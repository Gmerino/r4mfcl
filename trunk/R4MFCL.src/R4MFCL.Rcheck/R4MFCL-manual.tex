\nonstopmode{}
\documentclass[a4paper]{book}
\usepackage[times,inconsolata,hyper]{Rd}
\usepackage{makeidx}
\usepackage[utf8,latin1]{inputenc}
% \usepackage{graphicx} % @USE GRAPHICX@
\makeindex{}
\begin{document}
\chapter*{}
\begin{center}
{\textbf{\huge Package `R4MFCL'}}
\par\bigskip{\large \today}
\end{center}
\begin{description}
\raggedright{}
\item[Title]\AsIs{R functions for MULTIFAN-CL}
\item[Version]\AsIs{0.2}
\item[Author]\AsIs{Simon Hoyle, Pierre Kleiber, Shelton Harley, Nick Davies, and Adam Langley.}
\item[Description]\AsIs{Functions for automating the running and analysis of MULTIFAN-CL stock assessments, by manipulating the input files, and analyzing and plotting the output fles. R4MFCL is a collection of utility functions for stock assessments using the model MULTIFAN-CL (Fournier et al 1998; www.multifan-cl.org). There are several groups of R4MFCL functions: 1) input and output functions, for reading MULTIFAN-CL files into R objects and writing them back out as text files in the form that MULTIFAN-CL accepts as input. 2) data manipulation functions, for editing and restructuring the input objects. 3) plotting functions, for producing plots and maps from the result objects. 4) information functions, for comparing objects and giving information about, for example, flag settings.}
\item[Maintainer]\AsIs{Simon Hoyle }\email{simon.hoyle@gmail.com}\AsIs{ and Sam McKechnie
}\email{samm@spc.int}\AsIs{}
\item[License]\AsIs{GPL-2}
\item[Date]\AsIs{2013-09-10}
\item[Suggests]\AsIs{maps, mapdata, mapproj}
\end{description}
\Rdcontents{\R{} topics documented:}
\inputencoding{utf8}
\HeaderA{add.catch.frq}{add.catch.frq.Rd}{add.catch.frq}
\keyword{\textbackslash{}textasciitilde{}kwd1}{add.catch.frq}
\keyword{\textbackslash{}textasciitilde{}kwd2}{add.catch.frq}
%
\begin{Description}\relax

Used in sensitivity analyses, this function replaces the catch column relating to the fisheries specified - while accounting for the possibility that the number 
of rows or order differ between the sensitivity runs. 
The script assumes that the sensitivity *.frq will contain either: only those rows which are to be modified; or all rows to be modified. 
\end{Description}
%
\begin{Usage}
\begin{verbatim}
add.catch.frq(frq, filepath, fshries)
\end{verbatim}
\end{Usage}
%
\begin{Arguments}
\begin{ldescription}
\item[\code{frq}] 

The frq file object

\item[\code{filepath}] 

The path and filename of the frq file with the replacement fishery data. 

\item[\code{fshries}] 

The id numbers of the fisheries to be edited 

\end{ldescription}
\end{Arguments}
%
\begin{Author}\relax

Nick Davies
\end{Author}
%
\begin{Examples}
\begin{ExampleCode}
##---- Should be DIRECTLY executable !! ----
##-- ==>  Define data, use random,
##--	or do  help(data=index)  for the standard data sets.

\end{ExampleCode}
\end{Examples}
\inputencoding{utf8}
\HeaderA{add.cpue.frq}{add.cpue.frq.Rd}{add.cpue.frq}
\keyword{\textbackslash{}textasciitilde{}kwd1}{add.cpue.frq}
\keyword{\textbackslash{}textasciitilde{}kwd2}{add.cpue.frq}
%
\begin{Description}\relax
Replaces the nominal effort in the original .FRQ file with stanadrdised effort based on the CPUE index
Flexible to handle either sort of frq file and you have the choice to include the cv.  
Puts in -1 for effort first to make sure we account for missing values of CPUE 
\end{Description}
%
\begin{Usage}
\begin{verbatim}
add.cpue.frq(CPUE.file = "P:/yft/2009/Data Preparation/CPUE/indices/yft_JPstd_R1.txt", data = out.data, fishery = 1, add.cv = "T")
\end{verbatim}
\end{Usage}
%
\begin{Arguments}
\begin{ldescription}
\item[\code{CPUE.file}] 


\item[\code{data}] 


\item[\code{fishery}] 


\item[\code{add.cv}] 


\end{ldescription}
\end{Arguments}
%
\begin{Author}\relax

Shelton Harley
\end{Author}
%
\begin{Examples}
\begin{ExampleCode}
##---- Should be DIRECTLY executable !! ----
##-- ==>  Define data, use random,
##--	or do  help(data=index)  for the standard data sets.

\end{ExampleCode}
\end{Examples}
\inputencoding{utf8}
\HeaderA{add.flag}{add.flag.Rd}{add.flag}
\keyword{\textbackslash{}textasciitilde{}kwd1}{add.flag}
\keyword{\textbackslash{}textasciitilde{}kwd2}{add.flag}
%
\begin{Description}\relax

Adds a flag to the doitall object. 
\end{Description}
%
\begin{Usage}
\begin{verbatim}
add.flag(doitall, flagtype, flagnum, newval, phase)
\end{verbatim}
\end{Usage}
%
\begin{Arguments}
\begin{ldescription}
\item[\code{doitall}] 


\item[\code{flagtype}] 


\item[\code{flagnum}] 


\item[\code{newval}] 


\item[\code{phase}] 


\end{ldescription}
\end{Arguments}
%
\begin{Author}\relax

Simon Hoyle
\end{Author}
%
\begin{Examples}
\begin{ExampleCode}
##---- Should be DIRECTLY executable !! ----
##-- ==>  Define data, use random,
##--	or do  help(data=index)  for the standard data sets.

\end{ExampleCode}
\end{Examples}
\inputencoding{utf8}
\HeaderA{carry.effort.frq}{carry.effort.frq.Rd}{carry.effort.frq}
\keyword{\textbackslash{}textasciitilde{}kwd1}{carry.effort.frq}
\keyword{\textbackslash{}textasciitilde{}kwd2}{carry.effort.frq}
%
\begin{Description}\relax

Replaces the effort in the last year with effort in the previous year and sets catch to -1. 
\end{Description}
%
\begin{Usage}
\begin{verbatim}
carry.effort.frq(data = out.data, fishery = 1, last = 2008)
\end{verbatim}
\end{Usage}
%
\begin{Arguments}
\begin{ldescription}
\item[\code{data}] 


\item[\code{fishery}] 


\item[\code{last}] 


\end{ldescription}
\end{Arguments}
%
\begin{Author}\relax

Shelton Harley and Nick Davies
\end{Author}
%
\begin{Examples}
\begin{ExampleCode}
##---- Should be DIRECTLY executable !! ----
##-- ==>  Define data, use random,
##--	or do  help(data=index)  for the standard data sets.

\end{ExampleCode}
\end{Examples}
\inputencoding{utf8}
\HeaderA{change.fishflag}{change.fishflag.Rd}{change.fishflag}
\keyword{\textbackslash{}textasciitilde{}kwd1}{change.fishflag}
\keyword{\textbackslash{}textasciitilde{}kwd2}{change.fishflag}
%
\begin{Usage}
\begin{verbatim}
change.fishflag(a, fisheries, flagnum, newvals)
\end{verbatim}
\end{Usage}
%
\begin{Arguments}
\begin{ldescription}
\item[\code{a}] 


\item[\code{fisheries}] 


\item[\code{flagnum}] 


\item[\code{newvals}] 


\end{ldescription}
\end{Arguments}
%
\begin{Author}\relax
Simon Hoyle

\end{Author}
%
\begin{Examples}
\begin{ExampleCode}
##---- Should be DIRECTLY executable !! ----
##-- ==>  Define data, use random,
##--	or do  help(data=index)  for the standard data sets.

\end{ExampleCode}
\end{Examples}
\inputencoding{utf8}
\HeaderA{change.flag}{change.flag.Rd}{change.flag}
\keyword{\textbackslash{}textasciitilde{}kwd1}{change.flag}
\keyword{\textbackslash{}textasciitilde{}kwd2}{change.flag}
%
\begin{Usage}
\begin{verbatim}
change.flag(doitall, flagtype, flagnum, newval)
\end{verbatim}
\end{Usage}
%
\begin{Arguments}
\begin{ldescription}
\item[\code{doitall}] 


\item[\code{flagtype}] 


\item[\code{flagnum}] 


\item[\code{newval}] 


\end{ldescription}
\end{Arguments}
%
\begin{Author}\relax

Simon Hoyle
\end{Author}
%
\begin{Examples}
\begin{ExampleCode}
##---- Should be DIRECTLY executable !! ----
##-- ==>  Define data, use random,
##--	or do  help(data=index)  for the standard data sets.

\end{ExampleCode}
\end{Examples}
\inputencoding{utf8}
\HeaderA{change.negflag}{change.negflag.Rd}{change.negflag}
\keyword{\textbackslash{}textasciitilde{}kwd1}{change.negflag}
\keyword{\textbackslash{}textasciitilde{}kwd2}{change.negflag}
%
\begin{Usage}
\begin{verbatim}
change.negflag(doitall, flagtype, flagnum, newval)
\end{verbatim}
\end{Usage}
%
\begin{Arguments}
\begin{ldescription}
\item[\code{doitall}] 


\item[\code{flagtype}] 


\item[\code{flagnum}] 


\item[\code{newval}] 


\end{ldescription}
\end{Arguments}
%
\begin{Author}\relax
Simon Hoyle

\end{Author}
%
\begin{Examples}
\begin{ExampleCode}
##---- Should be DIRECTLY executable !! ----
##-- ==>  Define data, use random,
##--	or do  help(data=index)  for the standard data sets.

\end{ExampleCode}
\end{Examples}
\inputencoding{utf8}
\HeaderA{change.size.frq}{change.size.frq.Rd}{change.size.frq}
\keyword{\textbackslash{}textasciitilde{}kwd1}{change.size.frq}
\keyword{\textbackslash{}textasciitilde{}kwd2}{change.size.frq}
%
\begin{Usage}
\begin{verbatim}
change.size.frq(ver = 6, data = data, FISH = 1, LF.FILE = "P:/yft/2009/Data Preparation/size data/LLlendataR1.txt", WT.FILE = "P:/yft/2009/Data Preparation/size data/LLwtdataR1.txt")
\end{verbatim}
\end{Usage}
%
\begin{Arguments}
\begin{ldescription}
\item[\code{ver}] 


\item[\code{data}] 


\item[\code{FISH}] 


\item[\code{LF.FILE}] 


\item[\code{WT.FILE}] 


\end{ldescription}
\end{Arguments}
%
\begin{Author}\relax
Shelton Harley

\end{Author}
%
\begin{Examples}
\begin{ExampleCode}
##---- Should be DIRECTLY executable !! ----
##-- ==>  Define data, use random,
##--	or do  help(data=index)  for the standard data sets.

\end{ExampleCode}
\end{Examples}
\inputencoding{utf8}
\HeaderA{change\_data}{change\_data.Rd}{change.Rul.data}
\keyword{\textbackslash{}textasciitilde{}kwd1}{change\_data}
\keyword{\textbackslash{}textasciitilde{}kwd2}{change\_data}
%
\begin{Usage}
\begin{verbatim}
change_data(obj, searchtext, xlines, newline)
\end{verbatim}
\end{Usage}
%
\begin{Arguments}
\begin{ldescription}
\item[\code{obj}] 


\item[\code{searchtext}] 


\item[\code{xlines}] 


\item[\code{newline}] 


\end{ldescription}
\end{Arguments}
%
\begin{Author}\relax
Simon Hoyle

\end{Author}
%
\begin{Examples}
\begin{ExampleCode}
##---- Should be DIRECTLY executable !! ----
##-- ==>  Define data, use random,
##--	or do  help(data=index)  for the standard data sets.

\end{ExampleCode}
\end{Examples}
\inputencoding{utf8}
\HeaderA{check.eff.devs}{check.eff.devs.Rd}{check.eff.devs}
\keyword{\textbackslash{}textasciitilde{}kwd1}{check.eff.devs}
\keyword{\textbackslash{}textasciitilde{}kwd2}{check.eff.devs}
%
\begin{Usage}
\begin{verbatim}
check.eff.devs(parfile, repfile, frqfile, parlim = 5.9)
\end{verbatim}
\end{Usage}
%
\begin{Arguments}
\begin{ldescription}
\item[\code{parfile}] 


\item[\code{repfile}] 


\item[\code{frqfile}] 


\item[\code{parlim}] 


\end{ldescription}
\end{Arguments}
%
\begin{Author}\relax
Simon Hoyle

\end{Author}
%
\begin{Examples}
\begin{ExampleCode}
##---- Should be DIRECTLY executable !! ----
##-- ==>  Define data, use random,
##--	or do  help(data=index)  for the standard data sets.

\end{ExampleCode}
\end{Examples}
\inputencoding{utf8}
\HeaderA{check\_flag\_value}{check\_flag\_value.Rd}{check.Rul.flag.Rul.value}
\keyword{\textbackslash{}textasciitilde{}kwd1}{check\_flag\_value}
\keyword{\textbackslash{}textasciitilde{}kwd2}{check\_flag\_value}
%
\begin{Usage}
\begin{verbatim}
check_flag_value(parname, flagtype, flagnums, fishery = NA, flaglist = T)
\end{verbatim}
\end{Usage}
%
\begin{Arguments}
\begin{ldescription}
\item[\code{parname}] 


\item[\code{flagtype}] 


\item[\code{flagnums}] 


\item[\code{fishery}] 


\item[\code{flaglist}] 


\end{ldescription}
\end{Arguments}
%
\begin{Author}\relax
Simon Hoyle

\end{Author}
%
\begin{Examples}
\begin{ExampleCode}
##---- Should be DIRECTLY executable !! ----
##-- ==>  Define data, use random,
##--	or do  help(data=index)  for the standard data sets.

\end{ExampleCode}
\end{Examples}
\inputencoding{utf8}
\HeaderA{clean.lfdata}{clean.lfdata.Rd}{clean.lfdata}
\keyword{\textbackslash{}textasciitilde{}kwd1}{clean.lfdata}
\keyword{\textbackslash{}textasciitilde{}kwd2}{clean.lfdata}
%
\begin{Usage}
\begin{verbatim}
clean.lfdata(infrq)
\end{verbatim}
\end{Usage}
%
\begin{Arguments}
\begin{ldescription}
\item[\code{infrq}] 


\end{ldescription}
\end{Arguments}
%
\begin{Author}\relax
Simon Hoyle

\end{Author}
%
\begin{Examples}
\begin{ExampleCode}
##---- Should be DIRECTLY executable !! ----
##-- ==>  Define data, use random,
##--	or do  help(data=index)  for the standard data sets.

\end{ExampleCode}
\end{Examples}
\inputencoding{utf8}
\HeaderA{compare.ce.frq}{compare.ce.frq.Rd}{compare.ce.frq}
\keyword{\textbackslash{}textasciitilde{}kwd1}{compare.ce.frq}
\keyword{\textbackslash{}textasciitilde{}kwd2}{compare.ce.frq}
%
\begin{Usage}
\begin{verbatim}
compare.ce.frq(file1, file2, fm, plotname, fdesc = "")
\end{verbatim}
\end{Usage}
%
\begin{Arguments}
\begin{ldescription}
\item[\code{file1}] 


\item[\code{file2}] 


\item[\code{fm}] 


\item[\code{plotname}] 


\item[\code{fdesc}] 


\end{ldescription}
\end{Arguments}
%
\begin{Author}\relax
Adam Langley

\end{Author}
%
\begin{Examples}
\begin{ExampleCode}
##---- Should be DIRECTLY executable !! ----
##-- ==>  Define data, use random,
##--	or do  help(data=index)  for the standard data sets.

\end{ExampleCode}
\end{Examples}
\inputencoding{utf8}
\HeaderA{compare.frq}{compare.frq.Rd}{compare.frq}
\keyword{\textbackslash{}textasciitilde{}kwd1}{compare.frq}
\keyword{\textbackslash{}textasciitilde{}kwd2}{compare.frq}
%
\begin{Usage}
\begin{verbatim}
compare.frq(file1, file2, fm = "all", plotname, fdesc = "",lwd=2,what=rep(TRUE,3))
\end{verbatim}
\end{Usage}
%
\begin{Arguments}
\begin{ldescription}
\item[\code{file1}] 


\item[\code{file2}] 


\item[\code{fm}] 


\item[\code{plotname}] 


\item[\code{fdesc}] 


\item[\code{lwd}] 


\item[\code{what}] 


\end{ldescription}
\end{Arguments}
%
\begin{Author}\relax
Adam Langley and Simon Hoyle

\end{Author}
%
\begin{Examples}
\begin{ExampleCode}
##---- Should be DIRECTLY executable !! ----
##-- ==>  Define data, use random,
##--	or do  help(data=index)  for the standard data sets.

\end{ExampleCode}
\end{Examples}
\inputencoding{utf8}
\HeaderA{compare.size.frq}{compare.size.frq.Rd}{compare.size.frq}
\keyword{\textbackslash{}textasciitilde{}kwd1}{compare.size.frq}
\keyword{\textbackslash{}textasciitilde{}kwd2}{compare.size.frq}
%
\begin{Usage}
\begin{verbatim}
compare.size.frq(frq1, frq2, fishery = 5, wt=T, prefx = "_",doyears, fdesc="",summary=TRUE)
\end{verbatim}
\end{Usage}
%
\begin{Arguments}
\begin{ldescription}
\item[\code{frq1}] 


\item[\code{frq2}] 


\item[\code{fishery}] 


\item[\code{wt}] 


\item[\code{prefx}] 


\item[\code{doyears}] 


\item[\code{fdesc}] 


\item[\code{summary}] 


\end{ldescription}
\end{Arguments}
%
\begin{Author}\relax
Nick Davies

\end{Author}
%
\begin{Examples}
\begin{ExampleCode}
##---- Should be DIRECTLY executable !! ----
##-- ==>  Define data, use random,
##--	or do  help(data=index)  for the standard data sets.

\end{ExampleCode}
\end{Examples}
\inputencoding{utf8}
\HeaderA{compare\_par\_flags}{compare\_par\_flags.Rd}{compare.Rul.par.Rul.flags}
\keyword{\textbackslash{}textasciitilde{}kwd1}{compare\_par\_flags}
\keyword{\textbackslash{}textasciitilde{}kwd2}{compare\_par\_flags}
%
\begin{Description}\relax

Compares the flags in two par files and reports differences. 
\end{Description}
%
\begin{Usage}
\begin{verbatim}
compare_par_flags(par1, par2, flaglist = T)
\end{verbatim}
\end{Usage}
%
\begin{Arguments}
\begin{ldescription}
\item[\code{par1}] 


\item[\code{par2}] 


\item[\code{flaglist}] 


\end{ldescription}
\end{Arguments}
%
\begin{Author}\relax

Simon Hoyle
\end{Author}
%
\begin{Examples}
\begin{ExampleCode}
##---- Should be DIRECTLY executable !! ----
##-- ==>  Define data, use random,
##--	or do  help(data=index)  for the standard data sets.

\end{ExampleCode}
\end{Examples}
\inputencoding{utf8}
\HeaderA{condor.go}{condor.go.Rd}{condor.go}
\keyword{\textbackslash{}textasciitilde{}kwd1}{condor.go}
\keyword{\textbackslash{}textasciitilde{}kwd2}{condor.go}
%
\begin{Description}\relax

Used to compile MULTIFAN-CL files and submit a job to condor. 
\end{Description}
%
\begin{Usage}
\begin{verbatim}
condor.go(run.dir, frq.obj, tag.obj, doitall.obj, ini.obj, sub.obj, species = "alb", condor_f = condor_files, par.obj = NA, run_now = TRUE,fixpermissions=TRUE)
\end{verbatim}
\end{Usage}
%
\begin{Arguments}
\begin{ldescription}
\item[\code{run.dir}] 


\item[\code{frq.obj}] 


\item[\code{tag.obj}] 


\item[\code{doitall.obj}] 


\item[\code{ini.obj}] 


\item[\code{sub.obj}] 


\item[\code{species}] 


\item[\code{condor\_f}] 


\item[\code{par.obj}] 


\item[\code{run\_now}] 


\item[\code{fixpermissions}] 


\end{ldescription}
\end{Arguments}
%
\begin{Author}\relax
Simon Hoyle and Pierre Kleiber

\end{Author}
%
\begin{Examples}
\begin{ExampleCode}
##---- Should be DIRECTLY executable !! ----
##-- ==>  Define data, use random,
##--	or do  help(data=index)  for the standard data sets.

\end{ExampleCode}
\end{Examples}
\inputencoding{utf8}
\HeaderA{condor.go2}{condor.go2.Rd}{condor.go2}
\keyword{\textbackslash{}textasciitilde{}kwd1}{condor.go2}
\keyword{\textbackslash{}textasciitilde{}kwd2}{condor.go2}
%
\begin{Description}\relax

Used to compile MULTIFAN-CL files and submit a job to condor. 
\end{Description}
%
\begin{Usage}
\begin{verbatim}
condor.go2(run.dir, frq.obj, tag.obj, doitall.obj, ini.obj, sub.obj = suball, species = "alb", condor_f = condor_files, par.obj = NA, run_now = T)
\end{verbatim}
\end{Usage}
%
\begin{Arguments}
\begin{ldescription}
\item[\code{run.dir}] 


\item[\code{frq.obj}] 


\item[\code{tag.obj}] 


\item[\code{doitall.obj}] 


\item[\code{ini.obj}] 


\item[\code{sub.obj}] 


\item[\code{species}] 


\item[\code{condor\_f}] 


\item[\code{par.obj}] 


\item[\code{run\_now}] 


\end{ldescription}
\end{Arguments}
%
\begin{Author}\relax
Simon Hoyle

\end{Author}
%
\begin{Examples}
\begin{ExampleCode}
##---- Should be DIRECTLY executable !! ----
##-- ==>  Define data, use random,
##--	or do  help(data=index)  for the standard data sets.

\end{ExampleCode}
\end{Examples}
\inputencoding{utf8}
\HeaderA{convert.frq.ver6}{Convert.frq.ver6.Rd}{convert.frq.ver6}
\keyword{\textbackslash{}textasciitilde{}kwd1}{convert.frq.ver6}
\keyword{\textbackslash{}textasciitilde{}kwd2}{convert.frq.ver6}
%
\begin{Description}\relax

Converts a frq file frm vesion 5 to version 6. 
\end{Description}
%
\begin{Usage}
\begin{verbatim}
convert.frq.ver6(a)
\end{verbatim}
\end{Usage}
%
\begin{Arguments}
\begin{ldescription}
\item[\code{a}] 


\end{ldescription}
\end{Arguments}
%
\begin{Author}\relax

Simon Hoyle
\end{Author}
%
\begin{Examples}
\begin{ExampleCode}
##---- Should be DIRECTLY executable !! ----
##-- ==>  Define data, use random,
##--	or do  help(data=index)  for the standard data sets.

\end{ExampleCode}
\end{Examples}
\inputencoding{utf8}
\HeaderA{copy.condor.files}{copy.condor.files.Rd}{copy.condor.files}
\keyword{\textbackslash{}textasciitilde{}kwd1}{copy.condor.files}
\keyword{\textbackslash{}textasciitilde{}kwd2}{copy.condor.files}
%
\begin{Usage}
\begin{verbatim}
copy.condor.files(rundir, condor.files = "./condor.files/")
\end{verbatim}
\end{Usage}
%
\begin{Arguments}
\begin{ldescription}
\item[\code{rundir}] 


\item[\code{condor.files}] 


\end{ldescription}
\end{Arguments}
%
\begin{Author}\relax
Simon Hoyle and Pierre Kleiber

\end{Author}
%
\begin{Examples}
\begin{ExampleCode}
##---- Should be DIRECTLY executable !! ----
##-- ==>  Define data, use random,
##--	or do  help(data=index)  for the standard data sets.

\end{ExampleCode}
\end{Examples}
\inputencoding{utf8}
\HeaderA{create.missing.ce}{create.missing.ce.Rd}{create.missing.ce}
\keyword{\textbackslash{}textasciitilde{}kwd1}{create.missing.ce}
\keyword{\textbackslash{}textasciitilde{}kwd2}{create.missing.ce}
%
\begin{Usage}
\begin{verbatim}
create.missing.ce(data = data, yr = 2008, termfish)
\end{verbatim}
\end{Usage}
%
\begin{Arguments}
\begin{ldescription}
\item[\code{data}] 


\item[\code{yr}] 


\item[\code{termfish}] 


\end{ldescription}
\end{Arguments}
%
\begin{Author}\relax

Shelton Harley
\end{Author}
%
\begin{Examples}
\begin{ExampleCode}
##---- Should be DIRECTLY executable !! ----
##-- ==>  Define data, use random,
##--	or do  help(data=index)  for the standard data sets.

\end{ExampleCode}
\end{Examples}
\inputencoding{utf8}
\HeaderA{crit.fishery.summary}{crit.fishery.summary.Rd}{crit.fishery.summary}
\keyword{\textbackslash{}textasciitilde{}kwd1}{crit.fishery.summary}
\keyword{\textbackslash{}textasciitilde{}kwd2}{crit.fishery.summary}
%
\begin{Description}\relax

Takes the output from do.critical.calcs and gets the key reference points
\end{Description}
%
\begin{Usage}
\begin{verbatim}
crit.fishery.summary(crit)
\end{verbatim}
\end{Usage}
%
\begin{Arguments}
\begin{ldescription}
\item[\code{crit}] 


\end{ldescription}
\end{Arguments}
%
\begin{Author}\relax
Shelton Harley

\end{Author}
%
\begin{Examples}
\begin{ExampleCode}
##---- Should be DIRECTLY executable !! ----
##-- ==>  Define data, use random,
##--	or do  help(data=index)  for the standard data sets.

\end{ExampleCode}
\end{Examples}
\inputencoding{utf8}
\HeaderA{crit.summary}{crit.summary.Rd}{crit.summary}
\keyword{\textbackslash{}textasciitilde{}kwd1}{crit.summary}
\keyword{\textbackslash{}textasciitilde{}kwd2}{crit.summary}
%
\begin{Description}\relax

Takes the output from do.critical.calcs and gets the key reference points
\end{Description}
%
\begin{Usage}
\begin{verbatim}
crit.summary(crit, years)
\end{verbatim}
\end{Usage}
%
\begin{Arguments}
\begin{ldescription}
\item[\code{crit}] 


\item[\code{years}] 


\end{ldescription}
\end{Arguments}
%
\begin{Author}\relax
Shelton Harley

\end{Author}
%
\begin{Examples}
\begin{ExampleCode}
##---- Should be DIRECTLY executable !! ----
##-- ==>  Define data, use random,
##--	or do  help(data=index)  for the standard data sets.

\end{ExampleCode}
\end{Examples}
\inputencoding{utf8}
\HeaderA{datfromstr}{datfromstr.Rd}{datfromstr}
\keyword{\textbackslash{}textasciitilde{}kwd1}{datfromstr}
\keyword{\textbackslash{}textasciitilde{}kwd2}{datfromstr}
%
\begin{Usage}
\begin{verbatim}
datfromstr(datstring)
\end{verbatim}
\end{Usage}
%
\begin{Arguments}
\begin{ldescription}
\item[\code{datstring}] 


\end{ldescription}
\end{Arguments}
%
\begin{Author}\relax
Pierre Kleiber

\end{Author}
%
\begin{Examples}
\begin{ExampleCode}
##---- Should be DIRECTLY executable !! ----
##-- ==>  Define data, use random,
##--	or do  help(data=index)  for the standard data sets.

\end{ExampleCode}
\end{Examples}
\inputencoding{utf8}
\HeaderA{do.critical.calcs}{do.critical.calcs.Rd}{do.critical.calcs}
\keyword{\textbackslash{}textasciitilde{}kwd1}{do.critical.calcs}
\keyword{\textbackslash{}textasciitilde{}kwd2}{do.critical.calcs}
%
\begin{Description}\relax

Uses the dimensioning stuff provided in the rep file and the mean lengths and weights at age, 
and takes the fishery specific catch at age from the ests file
\end{Description}
%
\begin{Usage}
\begin{verbatim}
do.critical.calcs(repfile = "P:/yft/2007/BaseYFT/yftfinal2007.rep", ests = "P:/yft/2007/BaseYFT/ests.rep")
\end{verbatim}
\end{Usage}
%
\begin{Arguments}
\begin{ldescription}
\item[\code{repfile}] 


\item[\code{ests}] 


\end{ldescription}
\end{Arguments}
%
\begin{Author}\relax
Shelton Harley

\end{Author}
%
\begin{Examples}
\begin{ExampleCode}
##---- Should be DIRECTLY executable !! ----
##-- ==>  Define data, use random,
##--	or do  help(data=index)  for the standard data sets.

\end{ExampleCode}
\end{Examples}
\inputencoding{utf8}
\HeaderA{doit.rm\_flag}{doit.rm\_flag.Rd}{doit.rm.Rul.flag}
\keyword{\textbackslash{}textasciitilde{}kwd1}{doit.rm\_flag}
\keyword{\textbackslash{}textasciitilde{}kwd2}{doit.rm\_flag}
%
\begin{Usage}
\begin{verbatim}
doit.rm_flag(a, flagtype, flag, value)
\end{verbatim}
\end{Usage}
%
\begin{Arguments}
\begin{ldescription}
\item[\code{a}] 


\item[\code{flagtype}] 


\item[\code{flag}] 


\item[\code{value}] 


\end{ldescription}
\end{Arguments}
%
\begin{Author}\relax
Simon Hoyle

\end{Author}
%
\begin{Examples}
\begin{ExampleCode}
##---- Should be DIRECTLY executable !! ----
##-- ==>  Define data, use random,
##--	or do  help(data=index)  for the standard data sets.

\end{ExampleCode}
\end{Examples}
\inputencoding{utf8}
\HeaderA{effortcreep}{effortcreep.Rd}{effortcreep}
\keyword{\textbackslash{}textasciitilde{}kwd1}{effortcreep}
\keyword{\textbackslash{}textasciitilde{}kwd2}{effortcreep}
%
\begin{Description}\relax

Adjusts effort in a specified fishery at a consistent rate through time. 
\end{Description}
%
\begin{Usage}
\begin{verbatim}
effortcreep(frq.obj, fisheries, creep)
\end{verbatim}
\end{Usage}
%
\begin{Arguments}
\begin{ldescription}
\item[\code{frq.obj}] 


\item[\code{fisheries}] 


\item[\code{creep}] 


\end{ldescription}
\end{Arguments}
%
\begin{Author}\relax

Simon Hoyle
\end{Author}
%
\begin{Examples}
\begin{ExampleCode}
##---- Should be DIRECTLY executable !! ----
##-- ==>  Define data, use random,
##--	or do  help(data=index)  for the standard data sets.

\end{ExampleCode}
\end{Examples}
\inputencoding{utf8}
\HeaderA{fix\_growth}{fix\_growth.Rd}{fix.Rul.growth}
\keyword{\textbackslash{}textasciitilde{}kwd1}{fix\_growth}
\keyword{\textbackslash{}textasciitilde{}kwd2}{fix\_growth}
%
\begin{Usage}
\begin{verbatim}
fix_growth(a)
\end{verbatim}
\end{Usage}
%
\begin{Arguments}
\begin{ldescription}
\item[\code{a}] 


\end{ldescription}
\end{Arguments}
%
\begin{Author}\relax

Simon Hoyle
\end{Author}
%
\begin{Examples}
\begin{ExampleCode}
##---- Should be DIRECTLY executable !! ----
##-- ==>  Define data, use random,
##--	or do  help(data=index)  for the standard data sets.

\end{ExampleCode}
\end{Examples}
\inputencoding{utf8}
\HeaderA{frq.change.nint}{frq.change.nint.Rd}{frq.change.nint}
\keyword{\textbackslash{}textasciitilde{}kwd1}{frq.change.nint}
\keyword{\textbackslash{}textasciitilde{}kwd2}{frq.change.nint}
%
\begin{Usage}
\begin{verbatim}
frq.change.nint(in.frq, add.lfint, add.wfint)
\end{verbatim}
\end{Usage}
%
\begin{Arguments}
\begin{ldescription}
\item[\code{in.frq}] 


\item[\code{add.lfint}] 


\item[\code{add.wfint}] 


\end{ldescription}
\end{Arguments}
%
\begin{Author}\relax

Simon Hoyle
\end{Author}
%
\begin{Examples}
\begin{ExampleCode}
##---- Should be DIRECTLY executable !! ----
##-- ==>  Define data, use random,
##--	or do  help(data=index)  for the standard data sets.

\end{ExampleCode}
\end{Examples}
\inputencoding{utf8}
\HeaderA{frq.remove.size.or.weight.data}{frq.remove.size.or.weight.data.Rd}{frq.remove.size.or.weight.data}
\keyword{\textbackslash{}textasciitilde{}kwd1}{frq.remove.size.or.weight.data}
\keyword{\textbackslash{}textasciitilde{}kwd2}{frq.remove.size.or.weight.data}
%
\begin{Description}\relax

Provide a matrix (exclude) containing the following columns: Year | Month | week | fishery, and the function will remove the size and/or weight observations depending on the T/F flags. 
\end{Description}
%
\begin{Usage}
\begin{verbatim}
frq.remove.size.or.weight.data(data = test.data, exclude = exclude, size = T, weight = F)
\end{verbatim}
\end{Usage}
%
\begin{Arguments}
\begin{ldescription}
\item[\code{data}] 


\item[\code{exclude}] 


\item[\code{size}] 


\item[\code{weight}] 


\end{ldescription}
\end{Arguments}
%
\begin{Author}\relax

Shelton Harley
\end{Author}
%
\begin{Examples}
\begin{ExampleCode}
##---- Should be DIRECTLY executable !! ----
##-- ==>  Define data, use random,
##--	or do  help(data=index)  for the standard data sets.

\end{ExampleCode}
\end{Examples}
\inputencoding{utf8}
\HeaderA{get.critical.age}{get.critical.age.Rd}{get.critical.age}
\keyword{\textbackslash{}textasciitilde{}kwd1}{get.critical.age}
\keyword{\textbackslash{}textasciitilde{}kwd2}{get.critical.age}
%
\begin{Description}\relax

Calculates the age (and associated length, and weight) where the weight of a cohort is maximised. 
\end{Description}
%
\begin{Usage}
\begin{verbatim}
get.critical.age(data = Base.rep)
\end{verbatim}
\end{Usage}
%
\begin{Arguments}
\begin{ldescription}
\item[\code{data}] 

A .rep object

\end{ldescription}
\end{Arguments}
%
\begin{Author}\relax

Shelton Harley
\end{Author}
%
\begin{Examples}
\begin{ExampleCode}
##---- Should be DIRECTLY executable !! ----
##-- ==>  Define data, use random,
##--	or do  help(data=index)  for the standard data sets.

\end{ExampleCode}
\end{Examples}
\inputencoding{utf8}
\HeaderA{get.length.output}{get.length.output.Rd}{get.length.output}
\keyword{\textbackslash{}textasciitilde{}kwd1}{get.length.output}
\keyword{\textbackslash{}textasciitilde{}kwd2}{get.length.output}
%
\begin{Description}\relax

Not for standard use, and now may be obsolete. Adjusts the weight data to focus on the areas with most of the catch.
\end{Description}
%
\begin{Usage}
\begin{verbatim}
get.length.output(REGION = 1, DIR = "P:/yft/2009/Data Preparation/size data/")
\end{verbatim}
\end{Usage}
%
\begin{Arguments}
\begin{ldescription}
\item[\code{REGION}] 


\item[\code{DIR}] 


\end{ldescription}
\end{Arguments}
%
\begin{Author}\relax

Adam Langley
\end{Author}
%
\begin{Examples}
\begin{ExampleCode}
##---- Should be DIRECTLY executable !! ----
##-- ==>  Define data, use random,
##--	or do  help(data=index)  for the standard data sets.

\end{ExampleCode}
\end{Examples}
\inputencoding{utf8}
\HeaderA{get.outcomes}{get.outcomes.Rd}{get.outcomes}
\keyword{\textbackslash{}textasciitilde{}kwd1}{get.outcomes}
\keyword{\textbackslash{}textasciitilde{}kwd2}{get.outcomes}
%
\begin{Description}\relax

Extracts management information from result files
\end{Description}
%
\begin{Usage}
\begin{verbatim}
get.outcomes(file.rep, file.par, nofish = T)
\end{verbatim}
\end{Usage}
%
\begin{Arguments}
\begin{ldescription}
\item[\code{file.rep}] 


\item[\code{file.par}] 


\item[\code{nofish}] 


\end{ldescription}
\end{Arguments}
%
\begin{Author}\relax

Simon Hoyle
\end{Author}
%
\begin{Examples}
\begin{ExampleCode}
##---- Should be DIRECTLY executable !! ----
##-- ==>  Define data, use random,
##--	or do  help(data=index)  for the standard data sets.

\end{ExampleCode}
\end{Examples}
\inputencoding{utf8}
\HeaderA{get.tag.structure}{get.tag.structure.Rd}{get.tag.structure}
\keyword{\textbackslash{}textasciitilde{}kwd1}{get.tag.structure}
\keyword{\textbackslash{}textasciitilde{}kwd2}{get.tag.structure}
%
\begin{Description}\relax

Creates an object holding tag result information from the tag report file 
\end{Description}
%
\begin{Usage}
\begin{verbatim}
get.tag.structure(tagrepfile="temporary_tag_report",tagfile="skj.tag",year1=1972)
\end{verbatim}
\end{Usage}
%
\begin{Arguments}
\begin{ldescription}
\item[\code{tagrepfile}] 


\item[\code{tagfile}] 


\item[\code{year1}] 


\end{ldescription}
\end{Arguments}
%
\begin{Author}\relax
Pierre Kleiber

\end{Author}
%
\begin{Examples}
\begin{ExampleCode}
##---- Should be DIRECTLY executable !! ----
##-- ==>  Define data, use random,
##--	or do  help(data=index)  for the standard data sets.

\end{ExampleCode}
\end{Examples}
\inputencoding{utf8}
\HeaderA{get.weight.output}{get.weight.output.Rd}{get.weight.output}
\keyword{\textbackslash{}textasciitilde{}kwd1}{get.weight.output}
\keyword{\textbackslash{}textasciitilde{}kwd2}{get.weight.output}
%
\begin{Description}\relax

Not for standard use, and now may be obsolete. Adjusts the weight data to focus on the areas with most of the catch.
\end{Description}
%
\begin{Usage}
\begin{verbatim}
get.weight.output(REGION = 1, DIR = "P:/yft/2009/Data Preparation/size data/")
\end{verbatim}
\end{Usage}
%
\begin{Arguments}
\begin{ldescription}
\item[\code{REGION}] 


\item[\code{DIR}] 


\end{ldescription}
\end{Arguments}
%
\begin{Author}\relax

Adam Langley
\end{Author}
%
\begin{Examples}
\begin{ExampleCode}
##---- Should be DIRECTLY executable !! ----
##-- ==>  Define data, use random,
##--	or do  help(data=index)  for the standard data sets.

\end{ExampleCode}
\end{Examples}
\inputencoding{utf8}
\HeaderA{labels\_store}{labels\_store.Rd}{labels.Rul.store}
\keyword{\textbackslash{}textasciitilde{}kwd1}{labels\_store}
\keyword{\textbackslash{}textasciitilde{}kwd2}{labels\_store}
%
\begin{Description}\relax

Loads the labels.tmp file into an object. 
\end{Description}
%
\begin{Usage}
\begin{verbatim}
labels_store(labelfile = basecase.labels)
\end{verbatim}
\end{Usage}
%
\begin{Arguments}
\begin{ldescription}
\item[\code{labelfile}] 


\end{ldescription}
\end{Arguments}
%
\begin{Author}\relax

Simon Hoyle
\end{Author}
%
\begin{Examples}
\begin{ExampleCode}
##---- Should be DIRECTLY executable !! ----
##-- ==>  Define data, use random,
##--	or do  help(data=index)  for the standard data sets.

\end{ExampleCode}
\end{Examples}
\inputencoding{utf8}
\HeaderA{load.LFdata}{load.LFdata.Rd}{load.LFdata}
\keyword{\textbackslash{}textasciitilde{}kwd1}{load.LFdata}
\keyword{\textbackslash{}textasciitilde{}kwd2}{load.LFdata}
%
\begin{Description}\relax

Loads length frequency data from a database via ODBC and labels it with region. These data can then be processed into MFCL format, via another function.  
Mainly included as an example. 
\end{Description}
%
\begin{Usage}
\begin{verbatim}
load.LFdata(species = "ALB", gear = "L")
\end{verbatim}
\end{Usage}
%
\begin{Arguments}
\begin{ldescription}
\item[\code{species}] 


\item[\code{gear}] 


\end{ldescription}
\end{Arguments}
%
\begin{Author}\relax

Simon Hoyle
\end{Author}
%
\begin{Examples}
\begin{ExampleCode}
##---- Should be DIRECTLY executable !! ----
##-- ==>  Define data, use random,
##--	or do  help(data=index)  for the standard data sets.

\end{ExampleCode}
\end{Examples}
\inputencoding{utf8}
\HeaderA{make.projection.betyft.frq}{make.projection.betyft.frq.Rd}{make.projection.betyft.frq}
\keyword{\textbackslash{}textasciitilde{}kwd1}{make.projection.betyft.frq}
\keyword{\textbackslash{}textasciitilde{}kwd2}{make.projection.betyft.frq}
%
\begin{Description}\relax

Formats MULTIFAN-CL frq file for projections. Has been made obsolete by a collection of more complex projection functions. 
\end{Description}
%
\begin{Usage}
\begin{verbatim}
make.projection.betyft.frq(frq.in = base.frq, fish = 1:24, years = 10)
\end{verbatim}
\end{Usage}
%
\begin{Arguments}
\begin{ldescription}
\item[\code{frq.in}] 


\item[\code{fish}] 


\item[\code{years}] 


\end{ldescription}
\end{Arguments}
%
\begin{Author}\relax

Shelton Harley
\end{Author}
%
\begin{Examples}
\begin{ExampleCode}
##---- Should be DIRECTLY executable !! ----
##-- ==>  Define data, use random,
##--	or do  help(data=index)  for the standard data sets.

\end{ExampleCode}
\end{Examples}
\inputencoding{utf8}
\HeaderA{map\_all\_pacific}{map\_all\_pacific.Rd}{map.Rul.all.Rul.pacific}
\keyword{\textbackslash{}textasciitilde{}kwd1}{map\_all\_pacific}
\keyword{\textbackslash{}textasciitilde{}kwd2}{map\_all\_pacific}
%
\begin{Description}\relax

Draw a map with dimensions as specified, adding the EEZ boundaries. 
\end{Description}
%
\begin{Usage}
\begin{verbatim}
map_all_pacific(plot_title = "", lims = c(100, 300, -45, 45), eezfile = "L:/alb/2008/Pago/eznew2.txt")
\end{verbatim}
\end{Usage}
%
\begin{Arguments}
\begin{ldescription}
\item[\code{plot\_title}] 


\item[\code{lims}] 


\item[\code{eezfile}] 


\end{ldescription}
\end{Arguments}
%
\begin{Author}\relax

Adam Langley and Simon Hoyle
\end{Author}
%
\begin{Examples}
\begin{ExampleCode}
##---- Should be DIRECTLY executable !! ----
##-- ==>  Define data, use random,
##--	or do  help(data=index)  for the standard data sets.

\end{ExampleCode}
\end{Examples}
\inputencoding{utf8}
\HeaderA{merge.frq}{merge.frq.Rd}{merge.frq}
\keyword{\textbackslash{}textasciitilde{}kwd1}{merge.frq}
\keyword{\textbackslash{}textasciitilde{}kwd2}{merge.frq}
%
\begin{Description}\relax

Combine catch, effort and size frequency data when merging fisheries together. Currently set up for the albacore assessment and needs adapting. 
\end{Description}
%
\begin{Usage}
\begin{verbatim}
merge.frq(frq.obj, oldf, newf, mergelf = FALSE)
\end{verbatim}
\end{Usage}
%
\begin{Arguments}
\begin{ldescription}
\item[\code{frq.obj}] 


\item[\code{oldf}] 


\item[\code{newf}] 


\item[\code{mergelf}] 


\end{ldescription}
\end{Arguments}
%
\begin{Author}\relax

Simon Hoyle
\end{Author}
%
\begin{Examples}
\begin{ExampleCode}
##---- Should be DIRECTLY executable !! ----
##-- ==>  Define data, use random,
##--	or do  help(data=index)  for the standard data sets.

\end{ExampleCode}
\end{Examples}
\inputencoding{utf8}
\HeaderA{merge.tag}{merge.tag.Rd}{merge.tag}
\keyword{\textbackslash{}textasciitilde{}kwd1}{merge.tag}
\keyword{\textbackslash{}textasciitilde{}kwd2}{merge.tag}
%
\begin{Description}\relax

Change the fishery numbers for tag recoveries in a tag object. 
\end{Description}
%
\begin{Usage}
\begin{verbatim}
merge.tag(tag.obj, oldf, newf)
\end{verbatim}
\end{Usage}
%
\begin{Arguments}
\begin{ldescription}
\item[\code{tag.obj}] 


\item[\code{oldf}] 


\item[\code{newf}] 


\end{ldescription}
\end{Arguments}
%
\begin{Author}\relax

Simon Hoyle
\end{Author}
%
\begin{Examples}
\begin{ExampleCode}
##---- Should be DIRECTLY executable !! ----
##-- ==>  Define data, use random,
##--	or do  help(data=index)  for the standard data sets.

\end{ExampleCode}
\end{Examples}
\inputencoding{utf8}
\HeaderA{merge\_tag\_objs}{merge\_tag\_objs.Rd}{merge.Rul.tag.Rul.objs}
\keyword{\textbackslash{}textasciitilde{}kwd1}{merge\_tag\_objs}
\keyword{\textbackslash{}textasciitilde{}kwd2}{merge\_tag\_objs}
%
\begin{Description}\relax

Combine two tag objects into one. 
\end{Description}
%
\begin{Usage}
\begin{verbatim}
merge_tag_objs(obj1, obj2, relgrps)
\end{verbatim}
\end{Usage}
%
\begin{Arguments}
\begin{ldescription}
\item[\code{obj1}] 


\item[\code{obj2}] 


\item[\code{relgrps}] 


\end{ldescription}
\end{Arguments}
%
\begin{Author}\relax

Simon Hoyle
\end{Author}
%
\begin{Examples}
\begin{ExampleCode}
##---- Should be DIRECTLY executable !! ----
##-- ==>  Define data, use random,
##--	or do  help(data=index)  for the standard data sets.

\end{ExampleCode}
\end{Examples}
\inputencoding{utf8}
\HeaderA{pack.fisheries.frq}{pack.fisheries.frq.Rd}{pack.fisheries.frq}
\keyword{\textbackslash{}textasciitilde{}kwd1}{pack.fisheries.frq}
\keyword{\textbackslash{}textasciitilde{}kwd2}{pack.fisheries.frq}
%
\begin{Description}\relax

Remove gaps between fishery numbers
\end{Description}
%
\begin{Usage}
\begin{verbatim}
pack.fisheries.frq(frq.obj)
\end{verbatim}
\end{Usage}
%
\begin{Arguments}
\begin{ldescription}
\item[\code{frq.obj}] 


\end{ldescription}
\end{Arguments}
%
\begin{Author}\relax

Simon Hoyle
\end{Author}
%
\begin{Examples}
\begin{ExampleCode}
##---- Should be DIRECTLY executable !! ----
##-- ==>  Define data, use random,
##--	or do  help(data=index)  for the standard data sets.

\end{ExampleCode}
\end{Examples}
\inputencoding{utf8}
\HeaderA{plot.base.comparison}{plot.base.comparison.Rd}{plot.base.comparison}
\keyword{\textbackslash{}textasciitilde{}kwd1}{plot.base.comparison}
\keyword{\textbackslash{}textasciitilde{}kwd2}{plot.base.comparison}
%
\begin{Description}\relax

Plot F/FMSY against B/BMSY 
\end{Description}
%
\begin{Usage}
\begin{verbatim}
plot.base.comparison(baseres, labs)
\end{verbatim}
\end{Usage}
%
\begin{Arguments}
\begin{ldescription}
\item[\code{baseres}] 


\item[\code{labs}] 


\end{ldescription}
\end{Arguments}
%
\begin{Author}\relax

Adam Langley
\end{Author}
%
\begin{Examples}
\begin{ExampleCode}
##---- Should be DIRECTLY executable !! ----
##-- ==>  Define data, use random,
##--	or do  help(data=index)  for the standard data sets.

\end{ExampleCode}
\end{Examples}
\inputencoding{utf8}
\HeaderA{plot.biomass}{plot.biomass.Rd}{plot.biomass}
\keyword{\textbackslash{}textasciitilde{}kwd1}{plot.biomass}
\keyword{\textbackslash{}textasciitilde{}kwd2}{plot.biomass}
%
\begin{Description}\relax

Plots biomass by region and then combined
Option to add on CI as a polygoon
\end{Description}
%
\begin{Usage}
\begin{verbatim}
plot.biomass(plotdir = "H:/rmfcl/test/figs/", plotrep = test, varfile = NULL, type = "SSB", plotname = "H:/rmfcl/test/figs/biomass", plottype = "wmf")
\end{verbatim}
\end{Usage}
%
\begin{Arguments}
\begin{ldescription}
\item[\code{plotdir}] 


\item[\code{plotrep}] 


\item[\code{varfile}] 


\item[\code{type}] 


\item[\code{plotname}] 


\item[\code{plottype}] 


\end{ldescription}
\end{Arguments}
%
\begin{Author}\relax

Pierre Kleiber and Shelton Harley
\end{Author}
%
\begin{Examples}
\begin{ExampleCode}
##---- Should be DIRECTLY executable !! ----
##-- ==>  Define data, use random,
##--	or do  help(data=index)  for the standard data sets.

\end{ExampleCode}
\end{Examples}
\inputencoding{utf8}
\HeaderA{plot.biomass.combined}{plot.biomass.combined.Rd}{plot.biomass.combined}
\keyword{\textbackslash{}textasciitilde{}kwd1}{plot.biomass.combined}
\keyword{\textbackslash{}textasciitilde{}kwd2}{plot.biomass.combined}
%
\begin{Description}\relax

Plots biomass combined across all regions
Option to add on CI as a polygoon
\end{Description}
%
\begin{Usage}
\begin{verbatim}
plot.biomass.combined(plotdir = "H:/rmfcl/test/figs/", plotrep = test, varfile = NULL, type = "SSB", plotname = "H:/rmfcl/test/figs/biomass", plottype = "wmf")
\end{verbatim}
\end{Usage}
%
\begin{Arguments}
\begin{ldescription}
\item[\code{plotdir}] 


\item[\code{plotrep}] 


\item[\code{varfile}] 


\item[\code{type}] 


\item[\code{plotname}] 


\item[\code{plottype}] 


\end{ldescription}
\end{Arguments}
%
\begin{Author}\relax

Pierre Kleiber
\end{Author}
%
\begin{Examples}
\begin{ExampleCode}
##---- Should be DIRECTLY executable !! ----
##-- ==>  Define data, use random,
##--	or do  help(data=index)  for the standard data sets.

\end{ExampleCode}
\end{Examples}
\inputencoding{utf8}
\HeaderA{plot.F.time}{plot.F.time.Rd}{plot.F.time}
\keyword{\textbackslash{}textasciitilde{}kwd1}{plot.F.time}
\keyword{\textbackslash{}textasciitilde{}kwd2}{plot.F.time}
%
\begin{Description}\relax

Plots annual F by year for adults and juveniles (as defined by the maturity ogive in the *.ini file)
\end{Description}
%
\begin{Usage}
\begin{verbatim}
plot.F.time(plotdir = "H:/rmfcl/test/figs/", plotrep = "C:/assessments/alb/2008/6_area/28.splitgr3/plot-08.par.rep", inifile = "C:/assessments/alb/2008/6_area/28.splitgr3/alb.ini", plotname = "plotFtime", plottype = "wmf", COL = T)
\end{verbatim}
\end{Usage}
%
\begin{Arguments}
\begin{ldescription}
\item[\code{plotdir}] 


\item[\code{plotrep}] 


\item[\code{inifile}] 


\item[\code{plotname}] 


\item[\code{plottype}] 


\item[\code{COL}] 


\end{ldescription}
\end{Arguments}
%
\begin{Author}\relax

Pierre Kleiber
\end{Author}
%
\begin{Examples}
\begin{ExampleCode}
##---- Should be DIRECTLY executable !! ----
##-- ==>  Define data, use random,
##--	or do  help(data=index)  for the standard data sets.

\end{ExampleCode}
\end{Examples}
\inputencoding{utf8}
\HeaderA{plot.fishery.impact.r}{plot.fishery.impact.r.Rd}{plot.fishery.impact.r}
\keyword{\textbackslash{}textasciitilde{}kwd1}{plot.fishery.impact.r}
\keyword{\textbackslash{}textasciitilde{}kwd2}{plot.fishery.impact.r}
%
\begin{Description}\relax

Does the fishery impact plot by taking the output straight from the plot.rep files
You need to give the file names for the input files
\end{Description}
%
\begin{Usage}
\begin{verbatim}
plot.fishery.impact.r(plotdir = "H:/rmfcl/test/figs/", type = "Total", plotrep = testq0, impnames = c("ll", "psass", "psunass", "idph", "other"), plotname = "plotimpact", plottype = "wmf", COL = T)
\end{verbatim}
\end{Usage}
%
\begin{Arguments}
\begin{ldescription}
\item[\code{plotdir}] 


\item[\code{type}] 


\item[\code{plotrep}] 


\item[\code{impnames}] 


\item[\code{plotname}] 


\item[\code{plottype}] 


\item[\code{COL}] 


\end{ldescription}
\end{Arguments}
%
\begin{Author}\relax

Shelton Harley
\end{Author}
%
\begin{Examples}
\begin{ExampleCode}
##---- Should be DIRECTLY executable !! ----
##-- ==>  Define data, use random,
##--	or do  help(data=index)  for the standard data sets.

\end{ExampleCode}
\end{Examples}
\inputencoding{utf8}
\HeaderA{plot.Kobe}{plot.Kobe.Rd}{plot.Kobe}
\keyword{\textbackslash{}textasciitilde{}kwd1}{plot.Kobe}
\keyword{\textbackslash{}textasciitilde{}kwd2}{plot.Kobe}
%
\begin{Description}\relax

Take MULTIFAN-CL results and produce a Kobe plot
\end{Description}
%
\begin{Usage}
\begin{verbatim}
plot.Kobe(plotdir = "S:/OFP Publications/Tuna Fishery Assessment Report/2007/Figures/BET/", plotrep = test, type = "SSB", plotname = "Kobe", plottype = "wmf", COL = T)
\end{verbatim}
\end{Usage}
%
\begin{Arguments}
\begin{ldescription}
\item[\code{plotdir}] 


\item[\code{plotrep}] 


\item[\code{type}] 


\item[\code{plotname}] 


\item[\code{plottype}] 


\item[\code{COL}] 


\end{ldescription}
\end{Arguments}
%
\begin{Author}\relax

Adam Langley and Shelton Harley
\end{Author}
%
\begin{Examples}
\begin{ExampleCode}
##---- Should be DIRECTLY executable !! ----
##-- ==>  Define data, use random,
##--	or do  help(data=index)  for the standard data sets.

\end{ExampleCode}
\end{Examples}
\inputencoding{utf8}
\HeaderA{plot.Kobe.template.bw}{plot.Kobe.template.bw.Rd}{plot.Kobe.template.bw}
\keyword{\textbackslash{}textasciitilde{}kwd1}{plot.Kobe.template.bw}
\keyword{\textbackslash{}textasciitilde{}kwd2}{plot.Kobe.template.bw}
%
\begin{Description}\relax

Make the template for a Kobe plot without colour
\end{Description}
%
\begin{Usage}
\begin{verbatim}
plot.Kobe.template.bw(Type)
\end{verbatim}
\end{Usage}
%
\begin{Arguments}
\begin{ldescription}
\item[\code{Type}] 


\end{ldescription}
\end{Arguments}
%
\begin{Author}\relax

Adam Langley and Shelton Harley
\end{Author}
%
\begin{Examples}
\begin{ExampleCode}
##---- Should be DIRECTLY executable !! ----
##-- ==>  Define data, use random,
##--	or do  help(data=index)  for the standard data sets.

\end{ExampleCode}
\end{Examples}
\inputencoding{utf8}
\HeaderA{plot.Kobe.template.col}{plot.Kobe.template.col.Rd}{plot.Kobe.template.col}
\keyword{\textbackslash{}textasciitilde{}kwd1}{plot.Kobe.template.col}
\keyword{\textbackslash{}textasciitilde{}kwd2}{plot.Kobe.template.col}
%
\begin{Description}\relax

Make the template for a Kobe plot
\end{Description}
%
\begin{Usage}
\begin{verbatim}
plot.Kobe.template.col(Type)
\end{verbatim}
\end{Usage}
%
\begin{Arguments}
\begin{ldescription}
\item[\code{Type}] 


\end{ldescription}
\end{Arguments}
%
\begin{Author}\relax

Adam Langley and Shelton Harley
\end{Author}
%
\begin{Examples}
\begin{ExampleCode}
##---- Should be DIRECTLY executable !! ----
##-- ==>  Define data, use random,
##--	or do  help(data=index)  for the standard data sets.

\end{ExampleCode}
\end{Examples}
\inputencoding{utf8}
\HeaderA{plot.mfcl.betyft09}{plot.mfcl.betyft09.Rd}{plot.mfcl.betyft09}
\keyword{\textbackslash{}textasciitilde{}kwd1}{plot.mfcl.betyft09}
\keyword{\textbackslash{}textasciitilde{}kwd2}{plot.mfcl.betyft09}
%
\begin{Usage}
\begin{verbatim}
plot.mfcl.betyft09(lims = c(100, 260, -45, 45))
\end{verbatim}
\end{Usage}
%
\begin{Arguments}
\begin{ldescription}
\item[\code{lims}] 


\end{ldescription}
\end{Arguments}
%
\begin{Author}\relax

Adam Langley and Simon Hoyle
\end{Author}
%
\begin{Examples}
\begin{ExampleCode}
##---- Should be DIRECTLY executable !! ----
##-- ==>  Define data, use random,
##--	or do  help(data=index)  for the standard data sets.

\end{ExampleCode}
\end{Examples}
\inputencoding{utf8}
\HeaderA{plot.nofishing}{plot.nofishing.Rd}{plot.nofishing}
\keyword{\textbackslash{}textasciitilde{}kwd1}{plot.nofishing}
\keyword{\textbackslash{}textasciitilde{}kwd2}{plot.nofishing}
%
\begin{Usage}
\begin{verbatim}
plot.nofishing(plotdir = "H:/rmfcl/test/figs/", plotrep = testq0, type = "SSB", plotname = "H:/rmfcl/test/figs/Kobe", plottype = "wmf", COL = T)
\end{verbatim}
\end{Usage}
%
\begin{Arguments}
\begin{ldescription}
\item[\code{plotdir}] 


\item[\code{plotrep}] 


\item[\code{type}] 


\item[\code{plotname}] 


\item[\code{plottype}] 


\item[\code{COL}] 


\end{ldescription}
\end{Arguments}
%
\begin{Author}\relax

Shelton Harley
\end{Author}
%
\begin{Examples}
\begin{ExampleCode}
##---- Should be DIRECTLY executable !! ----
##-- ==>  Define data, use random,
##--	or do  help(data=index)  for the standard data sets.

\end{ExampleCode}
\end{Examples}
\inputencoding{utf8}
\HeaderA{plot.nofishing.combined}{plot.nofishing.combined.Rd}{plot.nofishing.combined}
\keyword{\textbackslash{}textasciitilde{}kwd1}{plot.nofishing.combined}
\keyword{\textbackslash{}textasciitilde{}kwd2}{plot.nofishing.combined}
%
\begin{Description}\relax

Plot the nofishing plots. 
\end{Description}
%
\begin{Usage}
\begin{verbatim}
plot.nofishing.combined(plotdir = "H:/rmfcl/test/figs/", plotrep = testq0, type = "SSB", plotname = "H:/rmfcl/test/figs/Kobe", plottype = "wmf", COL = T)
\end{verbatim}
\end{Usage}
%
\begin{Arguments}
\begin{ldescription}
\item[\code{plotdir}] 


\item[\code{plotrep}] 


\item[\code{type}] 


\item[\code{plotname}] 


\item[\code{plottype}] 


\item[\code{COL}] 


\end{ldescription}
\end{Arguments}
%
\begin{Author}\relax

Shelton Harley
\end{Author}
%
\begin{Examples}
\begin{ExampleCode}
##---- Should be DIRECTLY executable !! ----
##-- ==>  Define data, use random,
##--	or do  help(data=index)  for the standard data sets.

\end{ExampleCode}
\end{Examples}
\inputencoding{utf8}
\HeaderA{plot.pacific.alb}{plot.pacific.alb.Rd}{plot.pacific.alb}
\keyword{\textbackslash{}textasciitilde{}kwd1}{plot.pacific.alb}
\keyword{\textbackslash{}textasciitilde{}kwd2}{plot.pacific.alb}
%
\begin{Description}\relax

Plots the Pacific and includes boundaries for the albacore tuna model. 
\end{Description}
%
\begin{Usage}
\begin{verbatim}
plot.pacific.alb(plot_title = "", eez_dir = "I:/assessments/Pop dy modeling/MFCL/R functions/", plot_eez = T)
\end{verbatim}
\end{Usage}
%
\begin{Arguments}
\begin{ldescription}
\item[\code{plot\_title}] 


\item[\code{eez\_dir}] 


\item[\code{plot\_eez}] 


\end{ldescription}
\end{Arguments}
%
\begin{Author}\relax

Adam Langley and Simon Hoyle
\end{Author}
%
\begin{Examples}
\begin{ExampleCode}
##---- Should be DIRECTLY executable !! ----
##-- ==>  Define data, use random,
##--	or do  help(data=index)  for the standard data sets.

\end{ExampleCode}
\end{Examples}
\inputencoding{utf8}
\HeaderA{plot.pacific.skj}{plot.pacific.skj.Rd}{plot.pacific.skj}
\keyword{\textbackslash{}textasciitilde{}kwd1}{plot.pacific.skj}
\keyword{\textbackslash{}textasciitilde{}kwd2}{plot.pacific.skj}
%
\begin{Description}\relax

Plots the Pacific and includes boundaries for the skipjack tuna model. 
\end{Description}
%
\begin{Usage}
\begin{verbatim}
plot.pacific.skj(plot_title = "")
\end{verbatim}
\end{Usage}
%
\begin{Arguments}
\begin{ldescription}
\item[\code{plot\_title}] 


\end{ldescription}
\end{Arguments}
%
\begin{Author}\relax

Simon Hoyle
\end{Author}
%
\begin{Examples}
\begin{ExampleCode}
##---- Should be DIRECTLY executable !! ----
##-- ==>  Define data, use random,
##--	or do  help(data=index)  for the standard data sets.

\end{ExampleCode}
\end{Examples}
\inputencoding{utf8}
\HeaderA{plot.pacific.species}{plot.pacific.species.Rd}{plot.pacific.species}
\keyword{\textbackslash{}textasciitilde{}kwd1}{plot.pacific.species}
\keyword{\textbackslash{}textasciitilde{}kwd2}{plot.pacific.species}
%
\begin{Description}\relax

Plots the Pacific and includes boundaries for the specified model. 
\end{Description}
%
\begin{Usage}
\begin{verbatim}
plot.pacific.species(plot_title = "", uselims = NA, add.WCPFC = F, add.EPO = F, sp = "YFT", add.EEZ = T, eez_file = "I:/assessments/Pop dy modeling/MFCL/R functions/EZNEW2.TXT")
\end{verbatim}
\end{Usage}
%
\begin{Arguments}
\begin{ldescription}
\item[\code{plot\_title}] 


\item[\code{uselims}] 


\item[\code{add.WCPFC}] 


\item[\code{add.EPO}] 


\item[\code{sp}] 


\item[\code{add.EEZ}] 


\item[\code{eez\_file}] 


\end{ldescription}
\end{Arguments}
%
\begin{Author}\relax

Simon Hoyle
\end{Author}
%
\begin{Examples}
\begin{ExampleCode}
##---- Should be DIRECTLY executable !! ----
##-- ==>  Define data, use random,
##--	or do  help(data=index)  for the standard data sets.

\end{ExampleCode}
\end{Examples}
\inputencoding{utf8}
\HeaderA{plot.pacific.WCPFC}{plot.pacific.WCPFC.Rd}{plot.pacific.WCPFC}
\keyword{\textbackslash{}textasciitilde{}kwd1}{plot.pacific.WCPFC}
\keyword{\textbackslash{}textasciitilde{}kwd2}{plot.pacific.WCPFC}
%
\begin{Description}\relax

Plots the Pacific and includes boundaries for the yellowfin tuna model. 
\end{Description}
%
\begin{Usage}
\begin{verbatim}
plot.pacific.WCPFC(plot_title = "", lims = c(100, 260, -45, 45))
\end{verbatim}
\end{Usage}
%
\begin{Arguments}
\begin{ldescription}
\item[\code{plot\_title}] 


\item[\code{lims}] 


\end{ldescription}
\end{Arguments}
%
\begin{Author}\relax

Adam Langley and Simon Hoyle
\end{Author}
%
\begin{Examples}
\begin{ExampleCode}
##---- Should be DIRECTLY executable !! ----
##-- ==>  Define data, use random,
##--	or do  help(data=index)  for the standard data sets.

\end{ExampleCode}
\end{Examples}
\inputencoding{utf8}
\HeaderA{plot.pacific.yft}{plot.pacific.yft.Rd}{plot.pacific.yft}
\keyword{\textbackslash{}textasciitilde{}kwd1}{plot.pacific.yft}
\keyword{\textbackslash{}textasciitilde{}kwd2}{plot.pacific.yft}
%
\begin{Description}\relax

Plots the Pacific and includes boundaries for the yellowfin tuna model. 
\end{Description}
%
\begin{Usage}
\begin{verbatim}
plot.pacific.yft(plot_title = "", lims = c(100, 260, -45, 45), add.WCPFC = F)
\end{verbatim}
\end{Usage}
%
\begin{Arguments}
\begin{ldescription}
\item[\code{plot\_title}] 


\item[\code{lims}] 


\item[\code{add.WCPFC}] 


\end{ldescription}
\end{Arguments}
%
\begin{Author}\relax

Adam Langley and Simon Hoyle
\end{Author}
%
\begin{Examples}
\begin{ExampleCode}
##---- Should be DIRECTLY executable !! ----
##-- ==>  Define data, use random,
##--	or do  help(data=index)  for the standard data sets.

\end{ExampleCode}
\end{Examples}
\inputencoding{utf8}
\HeaderA{plot.recruitment}{plot.recruitment.Rd}{plot.recruitment}
\keyword{\textbackslash{}textasciitilde{}kwd1}{plot.recruitment}
\keyword{\textbackslash{}textasciitilde{}kwd2}{plot.recruitment}
%
\begin{Description}\relax

Plot recruitment by region and then combined
Option to add on CI as a polygoon for combined R only
\end{Description}
%
\begin{Usage}
\begin{verbatim}
plot.recruitment(plotdir = "H:/rmfcl/test/figs/", plotrep = test, varfile = NULL, plotname = "H:/rmfcl/test/figs/recruitment", plottype = "wmf")
\end{verbatim}
\end{Usage}
%
\begin{Arguments}
\begin{ldescription}
\item[\code{plotdir}] 


\item[\code{plotrep}] 


\item[\code{varfile}] 


\item[\code{plotname}] 


\item[\code{plottype}] 


\end{ldescription}
\end{Arguments}
%
\begin{Author}\relax

Pierre Kleiber
\end{Author}
%
\begin{Examples}
\begin{ExampleCode}
##---- Should be DIRECTLY executable !! ----
##-- ==>  Define data, use random,
##--	or do  help(data=index)  for the standard data sets.

\end{ExampleCode}
\end{Examples}
\inputencoding{utf8}
\HeaderA{plot.recruitment.combined}{plot.recruitment.combined.Rd}{plot.recruitment.combined}
\keyword{\textbackslash{}textasciitilde{}kwd1}{plot.recruitment.combined}
\keyword{\textbackslash{}textasciitilde{}kwd2}{plot.recruitment.combined}
%
\begin{Description}\relax

Plots biomass combined across all regions
Option to add on CI as a polygoon
\end{Description}
%
\begin{Usage}
\begin{verbatim}
plot.recruitment.combined(plotdir = "H:/rmfcl/test/figs/", plotrep = test, varfile = NULL, plotname = "H:/rmfcl/test/figs/recruitment_combined", plottype = "wmf")
\end{verbatim}
\end{Usage}
%
\begin{Arguments}
\begin{ldescription}
\item[\code{plotdir}] 


\item[\code{plotrep}] 


\item[\code{varfile}] 


\item[\code{plotname}] 


\item[\code{plottype}] 


\end{ldescription}
\end{Arguments}
%
\begin{Author}\relax

Pierre Kleiber
\end{Author}
%
\begin{Examples}
\begin{ExampleCode}
##---- Should be DIRECTLY executable !! ----
##-- ==>  Define data, use random,
##--	or do  help(data=index)  for the standard data sets.

\end{ExampleCode}
\end{Examples}
\inputencoding{utf8}
\HeaderA{plot\_cpue\_cv\_frq}{plot\_cpue\_cv\_frq.Rd}{plot.Rul.cpue.Rul.cv.Rul.frq}
\keyword{\textbackslash{}textasciitilde{}kwd1}{plot\_cpue\_cv\_frq}
\keyword{\textbackslash{}textasciitilde{}kwd2}{plot\_cpue\_cv\_frq}
%
\begin{Description}\relax

Takes a version 6 frq file and par file and plots the CPUE and CVs for the chosen fisheries. Currently the fisheries need to have effort wts.
\end{Description}
%
\begin{Usage}
\begin{verbatim}
plot_cpue_cv_frq(frq, parf, fisheries)
\end{verbatim}
\end{Usage}
%
\begin{Arguments}
\begin{ldescription}
\item[\code{frq}] 


\item[\code{parf}] 


\item[\code{fisheries}] 


\end{ldescription}
\end{Arguments}
%
\begin{Author}\relax

Simon Hoyle
\end{Author}
%
\begin{Examples}
\begin{ExampleCode}
##---- Should be DIRECTLY executable !! ----
##-- ==>  Define data, use random,
##--	or do  help(data=index)  for the standard data sets.

\end{ExampleCode}
\end{Examples}
\inputencoding{utf8}
\HeaderA{ppath}{ppath.Rd}{ppath}
\keyword{\textbackslash{}textasciitilde{}kwd1}{ppath}
\keyword{\textbackslash{}textasciitilde{}kwd2}{ppath}
%
\begin{Description}\relax

Joins parts of a file path together without fussing with "/" signs. 
\end{Description}
%
\begin{Usage}
\begin{verbatim}
ppath(p1,p2)
\end{verbatim}
\end{Usage}
%
\begin{Arguments}
\begin{ldescription}
\item[\code{p1}] 


\item[\code{p2}] 


\end{ldescription}
\end{Arguments}
%
\begin{Author}\relax
Pierre Kleiber

\end{Author}
%
\begin{Examples}
\begin{ExampleCode}
##---- Should be DIRECTLY executable !! ----
##-- ==>  Define data, use random,
##--	or do  help(data=index)  for the standard data sets.

\end{ExampleCode}
\end{Examples}
\inputencoding{utf8}
\HeaderA{R4MFCL}{R4MFCL-package.Rd}{R4MFCL}
%
\begin{Description}\relax

R4MFCL is a collection of utility functions for stock assessments using the model MULTIFAN-CL (Fournier et al 1998; www.multifan-cl.org). There are several groups of R4MFCL functions: 
- input and output functions, for reading MULTIFAN-CL files into R objects and writing them back out as text files in the form that MULTIFAN-CL accepts as input. 
- data manipulation functions, for editing and restructuring the input objects 
- plotting functions, for producing plots and maps from the result objects
- information functions, for comparing objects and giving information about, for example, flag settings. 
\end{Description}
\inputencoding{utf8}
\HeaderA{read.catchrep}{read.catchrep.Rd}{read.catchrep}
\keyword{\textbackslash{}textasciitilde{}kwd1}{read.catchrep}
\keyword{\textbackslash{}textasciitilde{}kwd2}{read.catchrep}
%
\begin{Description}\relax

Reads the catch.rep result file imnto an object. 
\end{Description}
%
\begin{Usage}
\begin{verbatim}
read.catchrep(catchrep.file)
\end{verbatim}
\end{Usage}
%
\begin{Arguments}
\begin{ldescription}
\item[\code{catchrep.file}] 


\end{ldescription}
\end{Arguments}
%
\begin{Author}\relax

Simon Hoyle
\end{Author}
%
\begin{Examples}
\begin{ExampleCode}
##---- Should be DIRECTLY executable !! ----
##-- ==>  Define data, use random,
##--	or do  help(data=index)  for the standard data sets.

\end{ExampleCode}
\end{Examples}
\inputencoding{utf8}
\HeaderA{read.ests}{read.ests.Rd}{read.ests}
\keyword{\textbackslash{}textasciitilde{}kwd1}{read.ests}
\keyword{\textbackslash{}textasciitilde{}kwd2}{read.ests}
%
\begin{Description}\relax

Load the ests.rep file into an object. 
\end{Description}
%
\begin{Usage}
\begin{verbatim}
read.ests(rep.obj, ests = "C:/assessments/alb/2008/6_area/28.splitgr3/ests.rep", x = 1)
\end{verbatim}
\end{Usage}
%
\begin{Arguments}
\begin{ldescription}
\item[\code{rep.obj}] 


\item[\code{ests}] 


\item[\code{x}] 


\end{ldescription}
\end{Arguments}
%
\begin{Author}\relax

Simon Hoyle
\end{Author}
%
\begin{Examples}
\begin{ExampleCode}
##---- Should be DIRECTLY executable !! ----
##-- ==>  Define data, use random,
##--	or do  help(data=index)  for the standard data sets.

\end{ExampleCode}
\end{Examples}
\inputencoding{utf8}
\HeaderA{read.fit}{read.fit.Rd}{read.fit}
\keyword{\textbackslash{}textasciitilde{}kwd1}{read.fit}
\keyword{\textbackslash{}textasciitilde{}kwd2}{read.fit}
%
\begin{Description}\relax

Loads the observed and expected size frequency from the *.fit file by fishery and time period. 
\end{Description}
%
\begin{Usage}
\begin{verbatim}
read.fit(fit.file)
\end{verbatim}
\end{Usage}
%
\begin{Arguments}
\begin{ldescription}
\item[\code{fit.file}] 


\end{ldescription}
\end{Arguments}
%
\begin{Author}\relax

Simon Hoyle
\end{Author}
%
\begin{Examples}
\begin{ExampleCode}
##---- Should be DIRECTLY executable !! ----
##-- ==>  Define data, use random,
##--	or do  help(data=index)  for the standard data sets.

\end{ExampleCode}
\end{Examples}
\inputencoding{utf8}
\HeaderA{read.frq}{read.frq.Rd}{read.frq}
\keyword{\textbackslash{}textasciitilde{}kwd1}{read.frq}
\keyword{\textbackslash{}textasciitilde{}kwd2}{read.frq}
%
\begin{Description}\relax

Reads in the frq file into a frq object for either version 4 or 6+. 
\end{Description}
%
\begin{Usage}
\begin{verbatim}
read.frq(frq.file, frq.title = "", ntop = 0, fishdefs = NA)
\end{verbatim}
\end{Usage}
%
\begin{Arguments}
\begin{ldescription}
\item[\code{frq.file}] 


\item[\code{frq.title}] 


\item[\code{ntop}] 


\item[\code{fishdefs}] 


\end{ldescription}
\end{Arguments}
%
\begin{Author}\relax

Simon Hoyle
\end{Author}
%
\begin{Examples}
\begin{ExampleCode}
##---- Should be DIRECTLY executable !! ----
##-- ==>  Define data, use random,
##--	or do  help(data=index)  for the standard data sets.

\end{ExampleCode}
\end{Examples}
\inputencoding{utf8}
\HeaderA{read.impact}{read.impact.Rd}{read.impact}
\keyword{\textbackslash{}textasciitilde{}kwd1}{read.impact}
\keyword{\textbackslash{}textasciitilde{}kwd2}{read.impact}
%
\begin{Description}\relax

Reads the *.rep files from various impact runs into their own objects, and names them. 
\end{Description}
%
\begin{Usage}
\begin{verbatim}
read.impact(impdir = "H:/rmfcl/test/", impnames = c("ll", "psassoc", "psunassoc", "idph", "other"))
\end{verbatim}
\end{Usage}
%
\begin{Arguments}
\begin{ldescription}
\item[\code{impdir}] 


\item[\code{impnames}] 


\end{ldescription}
\end{Arguments}
%
\begin{Author}\relax

Shelton Harley
\end{Author}
%
\begin{Examples}
\begin{ExampleCode}
##---- Should be DIRECTLY executable !! ----
##-- ==>  Define data, use random,
##--	or do  help(data=index)  for the standard data sets.

\end{ExampleCode}
\end{Examples}
\inputencoding{utf8}
\HeaderA{read.ini}{read.ini.Rd}{read.ini}
\keyword{\textbackslash{}textasciitilde{}kwd1}{read.ini}
\keyword{\textbackslash{}textasciitilde{}kwd2}{read.ini}
%
\begin{Description}\relax

Reads the *.ini data input file into an object. 
\end{Description}
%
\begin{Usage}
\begin{verbatim}
read.ini(ini.file)
\end{verbatim}
\end{Usage}
%
\begin{Arguments}
\begin{ldescription}
\item[\code{ini.file}] 


\end{ldescription}
\end{Arguments}
%
\begin{Author}\relax

Simon Hoyle
\end{Author}
%
\begin{Examples}
\begin{ExampleCode}
##---- Should be DIRECTLY executable !! ----
##-- ==>  Define data, use random,
##--	or do  help(data=index)  for the standard data sets.

\end{ExampleCode}
\end{Examples}
\inputencoding{utf8}
\HeaderA{read.par}{read.par.Rd}{read.par}
\keyword{\textbackslash{}textasciitilde{}kwd1}{read.par}
\keyword{\textbackslash{}textasciitilde{}kwd2}{read.par}
%
\begin{Description}\relax

Reads the *.par output and input MULTIFAN-CL parameter file into an object. 
\end{Description}
%
\begin{Usage}
\begin{verbatim}
read.par(par.file)
\end{verbatim}
\end{Usage}
%
\begin{Arguments}
\begin{ldescription}
\item[\code{par.file}] 


\end{ldescription}
\end{Arguments}
%
\begin{Author}\relax

Simon Hoyle
\end{Author}
%
\begin{Examples}
\begin{ExampleCode}
##---- Should be DIRECTLY executable !! ----
##-- ==>  Define data, use random,
##--	or do  help(data=index)  for the standard data sets.

\end{ExampleCode}
\end{Examples}
\inputencoding{utf8}
\HeaderA{read.rep}{read.rep.Rd}{read.rep}
\keyword{\textbackslash{}textasciitilde{}kwd1}{read.rep}
\keyword{\textbackslash{}textasciitilde{}kwd2}{read.rep}
%
\begin{Description}\relax

Reads the rep file, which contains most of the important results, into an object. 
\end{Description}
%
\begin{Usage}
\begin{verbatim}
read.rep(rep.file)
\end{verbatim}
\end{Usage}
%
\begin{Arguments}
\begin{ldescription}
\item[\code{rep.file}] 


\end{ldescription}
\end{Arguments}
%
\begin{Author}\relax

Simon Hoyle
\end{Author}
%
\begin{Examples}
\begin{ExampleCode}
##---- Should be DIRECTLY executable !! ----
##-- ==>  Define data, use random,
##--	or do  help(data=index)  for the standard data sets.

\end{ExampleCode}
\end{Examples}
\inputencoding{utf8}
\HeaderA{read.tag}{read.tag.Rd}{read.tag}
\keyword{\textbackslash{}textasciitilde{}kwd1}{read.tag}
\keyword{\textbackslash{}textasciitilde{}kwd2}{read.tag}
%
\begin{Description}\relax

Reads the *.tag data input file into an object. 
\end{Description}
%
\begin{Usage}
\begin{verbatim}
read.tag(tagfile)
\end{verbatim}
\end{Usage}
%
\begin{Arguments}
\begin{ldescription}
\item[\code{tagfile}] 


\end{ldescription}
\end{Arguments}
%
\begin{Author}\relax

Simon Hoyle
\end{Author}
%
\begin{Examples}
\begin{ExampleCode}
##---- Should be DIRECTLY executable !! ----
##-- ==>  Define data, use random,
##--	or do  help(data=index)  for the standard data sets.

\end{ExampleCode}
\end{Examples}
\inputencoding{utf8}
\HeaderA{read.var}{read.var.Rd}{read.var}
\keyword{\textbackslash{}textasciitilde{}kwd1}{read.var}
\keyword{\textbackslash{}textasciitilde{}kwd2}{read.var}
%
\begin{Description}\relax

Reads the *.var result file into an object. 
\end{Description}
%
\begin{Usage}
\begin{verbatim}
read.var(var.file)
\end{verbatim}
\end{Usage}
%
\begin{Arguments}
\begin{ldescription}
\item[\code{var.file}] 


\end{ldescription}
\end{Arguments}
%
\begin{Author}\relax

Simon Hoyle
\end{Author}
%
\begin{Examples}
\begin{ExampleCode}
##---- Should be DIRECTLY executable !! ----
##-- ==>  Define data, use random,
##--	or do  help(data=index)  for the standard data sets.

\end{ExampleCode}
\end{Examples}
\inputencoding{utf8}
\HeaderA{read\_nmd.frq}{read\_nmd.frq.Rd}{read.Rul.nmd.frq}
\keyword{\textbackslash{}textasciitilde{}kwd1}{read\_nmd.frq}
\keyword{\textbackslash{}textasciitilde{}kwd2}{read\_nmd.frq}
%
\begin{Description}\relax

Reads in the frq file into a frq object for either version 4 or 6+. 
\end{Description}
%
\begin{Usage}
\begin{verbatim}
read_nmd.frq(frq.file, frq.title = "", ntop = 0, fishdefs = NA)
\end{verbatim}
\end{Usage}
%
\begin{Arguments}
\begin{ldescription}
\item[\code{frq.file}] 


\item[\code{frq.title}] 


\item[\code{ntop}] 


\item[\code{fishdefs}] 


\end{ldescription}
\end{Arguments}
%
\begin{Author}\relax

Simon Hoyle
\end{Author}
%
\begin{Examples}
\begin{ExampleCode}
##---- Should be DIRECTLY executable !! ----
##-- ==>  Define data, use random,
##--	or do  help(data=index)  for the standard data sets.

\end{ExampleCode}
\end{Examples}
\inputencoding{utf8}
\HeaderA{reconstruct.frq.ce}{reconstruct.frq.ce.Rd}{reconstruct.frq.ce}
\keyword{\textbackslash{}textasciitilde{}kwd1}{reconstruct.frq.ce}
\keyword{\textbackslash{}textasciitilde{}kwd2}{reconstruct.frq.ce}
%
\begin{Description}\relax

Replaces the nominal effort in the original .FRQ file with stanadrdised effort based on a CPUE index. Not generalised - specific to bigeye 2008. 

\end{Description}
%
\begin{Usage}
\begin{verbatim}
reconstruct.frq.ce(CPUE.file = "X:/yft/2009/Data Preparation/CPUE/indices/yft_JPstd_R1.txt", data = out.data, fishery = 1)
\end{verbatim}
\end{Usage}
%
\begin{Arguments}
\begin{ldescription}
\item[\code{CPUE.file}] 


\item[\code{data}] 


\item[\code{fishery}] 


\end{ldescription}
\end{Arguments}
%
\begin{Author}\relax

Shelton Harley
\end{Author}
%
\begin{Examples}
\begin{ExampleCode}
##---- Should be DIRECTLY executable !! ----
##-- ==>  Define data, use random,
##--	or do  help(data=index)  for the standard data sets.

\end{ExampleCode}
\end{Examples}
\inputencoding{utf8}
\HeaderA{reconstruct.frq.size}{reconstruct.frq.size.Rd}{reconstruct.frq.size}
\keyword{\textbackslash{}textasciitilde{}kwd1}{reconstruct.frq.size}
\keyword{\textbackslash{}textasciitilde{}kwd2}{reconstruct.frq.size}
%
\begin{Description}\relax

Pull in new size and weight frequency data and rebuild the frq object. Not generalized - specific to WCPO bigeye. 
\end{Description}
%
\begin{Usage}
\begin{verbatim}
reconstruct.frq.size(data = data, FISH = 1, LF.FILE = "P:/yft/2009/Data Preparation/size data/LLlendataR1.txt", WT.FILE = "P:/yft/2009/Data Preparation/size data/LLwtdataR1.txt")
\end{verbatim}
\end{Usage}
%
\begin{Arguments}
\begin{ldescription}
\item[\code{data}] 


\item[\code{FISH}] 


\item[\code{LF.FILE}] 


\item[\code{WT.FILE}] 


\end{ldescription}
\end{Arguments}
%
\begin{Author}\relax

Shelton Harley
\end{Author}
%
\begin{Examples}
\begin{ExampleCode}
##---- Should be DIRECTLY executable !! ----
##-- ==>  Define data, use random,
##--	or do  help(data=index)  for the standard data sets.

\end{ExampleCode}
\end{Examples}
\inputencoding{utf8}
\HeaderA{region\_single\_frq}{region\_single\_frq.Rd}{region.Rul.single.Rul.frq}
\keyword{\textbackslash{}textasciitilde{}kwd1}{region\_single\_frq}
\keyword{\textbackslash{}textasciitilde{}kwd2}{region\_single\_frq}
%
\begin{Description}\relax

Change a frq object to a single region, removing all fisheries outside that region
\end{Description}
%
\begin{Usage}
\begin{verbatim}
region_single_frq(frq, region)
\end{verbatim}
\end{Usage}
%
\begin{Arguments}
\begin{ldescription}
\item[\code{frq}] 


\item[\code{region}] 


\end{ldescription}
\end{Arguments}
%
\begin{Author}\relax

Simon Hoyle
\end{Author}
%
\begin{Examples}
\begin{ExampleCode}
##---- Should be DIRECTLY executable !! ----
##-- ==>  Define data, use random,
##--	or do  help(data=index)  for the standard data sets.

\end{ExampleCode}
\end{Examples}
\inputencoding{utf8}
\HeaderA{region\_single\_ini}{region\_single\_ini.Rd}{region.Rul.single.Rul.ini}
\keyword{\textbackslash{}textasciitilde{}kwd1}{region\_single\_ini}
\keyword{\textbackslash{}textasciitilde{}kwd2}{region\_single\_ini}
%
\begin{Description}\relax

Change an ini object to a single region, removing all fisheries outside that region
\end{Description}
%
\begin{Usage}
\begin{verbatim}
region_single_ini(ini)
\end{verbatim}
\end{Usage}
%
\begin{Arguments}
\begin{ldescription}
\item[\code{ini}] 


\end{ldescription}
\end{Arguments}
%
\begin{Author}\relax

Simon Hoyle
\end{Author}
%
\begin{Examples}
\begin{ExampleCode}
##---- Should be DIRECTLY executable !! ----
##-- ==>  Define data, use random,
##--	or do  help(data=index)  for the standard data sets.

\end{ExampleCode}
\end{Examples}
\inputencoding{utf8}
\HeaderA{region\_single\_tag}{region\_single\_tag.Rd}{region.Rul.single.Rul.tag}
\keyword{\textbackslash{}textasciitilde{}kwd1}{region\_single\_tag}
\keyword{\textbackslash{}textasciitilde{}kwd2}{region\_single\_tag}
%
\begin{Description}\relax

Change a tag object to a single region, removing all fisheries outside that region
\end{Description}
%
\begin{Usage}
\begin{verbatim}
region_single_tag(tag, region, keepfish)
\end{verbatim}
\end{Usage}
%
\begin{Arguments}
\begin{ldescription}
\item[\code{tag}] 


\item[\code{region}] 


\item[\code{keepfish}] 


\end{ldescription}
\end{Arguments}
%
\begin{Author}\relax

Simon Hoyle
\end{Author}
%
\begin{Examples}
\begin{ExampleCode}
##---- Should be DIRECTLY executable !! ----
##-- ==>  Define data, use random,
##--	or do  help(data=index)  for the standard data sets.

\end{ExampleCode}
\end{Examples}
\inputencoding{utf8}
\HeaderA{regroup\_fishery\_grps.doitall}{regroup\_fishery\_grps.doitall.Rd}{regroup.Rul.fishery.Rul.grps.doitall}
\keyword{\textbackslash{}textasciitilde{}kwd1}{regroup\_fishery\_grps.doitall}
\keyword{\textbackslash{}textasciitilde{}kwd2}{regroup\_fishery\_grps.doitall}
%
\begin{Description}\relax

Regroup all the fisheries in the vector f to the groups in the vector newgrps for the specified flag. 
\end{Description}
%
\begin{Usage}
\begin{verbatim}
regroup_fishery_grps.doitall(doitall, f, flag, newgrps)
\end{verbatim}
\end{Usage}
%
\begin{Arguments}
\begin{ldescription}
\item[\code{doitall}] 


\item[\code{f}] 


\item[\code{flag}] 


\item[\code{newgrps}] 


\end{ldescription}
\end{Arguments}
%
\begin{Author}\relax

Simon Hoyle
\end{Author}
%
\begin{Examples}
\begin{ExampleCode}
##---- Should be DIRECTLY executable !! ----
##-- ==>  Define data, use random,
##--	or do  help(data=index)  for the standard data sets.

\end{ExampleCode}
\end{Examples}
\inputencoding{utf8}
\HeaderA{rename.fisheries.doitall}{rename.fisheries.doitall.Rd}{rename.fisheries.doitall}
\keyword{\textbackslash{}textasciitilde{}kwd1}{rename.fisheries.doitall}
\keyword{\textbackslash{}textasciitilde{}kwd2}{rename.fisheries.doitall}
%
\begin{Description}\relax

Rename all the fisheries in the vector oldfs to the numbers in the vector newfs. 
\end{Description}
%
\begin{Usage}
\begin{verbatim}
rename.fisheries.doitall(doitall, oldfs, newfs)
\end{verbatim}
\end{Usage}
%
\begin{Arguments}
\begin{ldescription}
\item[\code{doitall}] 


\item[\code{oldfs}] 


\item[\code{newfs}] 


\end{ldescription}
\end{Arguments}
%
\begin{Author}\relax

Simon Hoyle
\end{Author}
%
\begin{Examples}
\begin{ExampleCode}
##---- Should be DIRECTLY executable !! ----
##-- ==>  Define data, use random,
##--	or do  help(data=index)  for the standard data sets.

\end{ExampleCode}
\end{Examples}
\inputencoding{utf8}
\HeaderA{rename.fisheries.frq}{rename.fisheries.frq.Rd}{rename.fisheries.frq}
\keyword{\textbackslash{}textasciitilde{}kwd1}{rename.fisheries.frq}
\keyword{\textbackslash{}textasciitilde{}kwd2}{rename.fisheries.frq}
%
\begin{Description}\relax

Rename all the fisheries in the vector oldfish to the numbers in the vector newfish. 
\end{Description}
%
\begin{Usage}
\begin{verbatim}
rename.fisheries.frq(frq.obj, oldfish, newfish)
\end{verbatim}
\end{Usage}
%
\begin{Arguments}
\begin{ldescription}
\item[\code{frq.obj}] 


\item[\code{oldfish}] 


\item[\code{newfish}] 


\end{ldescription}
\end{Arguments}
%
\begin{Author}\relax

Simon Hoyle
\end{Author}
%
\begin{Examples}
\begin{ExampleCode}
##---- Should be DIRECTLY executable !! ----
##-- ==>  Define data, use random,
##--	or do  help(data=index)  for the standard data sets.

\end{ExampleCode}
\end{Examples}
\inputencoding{utf8}
\HeaderA{rename.fisheries.tag}{rename.fisheries.tag.Rd}{rename.fisheries.tag}
\keyword{\textbackslash{}textasciitilde{}kwd1}{rename.fisheries.tag}
\keyword{\textbackslash{}textasciitilde{}kwd2}{rename.fisheries.tag}
%
\begin{Description}\relax

Rename the fisheries in oldfish to the fishery numbers in newfish. 
\end{Description}
%
\begin{Usage}
\begin{verbatim}
rename.fisheries.tag(tag.obj, oldfish, newfish)
\end{verbatim}
\end{Usage}
%
\begin{Arguments}
\begin{ldescription}
\item[\code{tag.obj}] 


\item[\code{oldfish}] 


\item[\code{newfish}] 


\end{ldescription}
\end{Arguments}
%
\begin{Author}\relax

Simon Hoyle
\end{Author}
%
\begin{Examples}
\begin{ExampleCode}
##---- Should be DIRECTLY executable !! ----
##-- ==>  Define data, use random,
##--	or do  help(data=index)  for the standard data sets.

\end{ExampleCode}
\end{Examples}
\inputencoding{utf8}
\HeaderA{rename.fishery.grps.doitall}{rename.fishery.grps.doitall.Rd}{rename.fishery.grps.doitall}
\keyword{\textbackslash{}textasciitilde{}kwd1}{rename.fishery.grps.doitall}
\keyword{\textbackslash{}textasciitilde{}kwd2}{rename.fishery.grps.doitall}
%
\begin{Description}\relax

Rename all the fisheries in the vector oldfs to the numbers in the vector newfs, for the specified flag. 
\end{Description}
%
\begin{Usage}
\begin{verbatim}
rename.fishery.grps.doitall(doitall, oldfs, newfs, flag, keep = T, newgrps = c(0))
\end{verbatim}
\end{Usage}
%
\begin{Arguments}
\begin{ldescription}
\item[\code{doitall}] 


\item[\code{oldfs}] 


\item[\code{newfs}] 


\item[\code{flag}] 


\item[\code{keep}] 


\item[\code{newgrps}] 


\end{ldescription}
\end{Arguments}
%
\begin{Author}\relax

Simon Hoyle
\end{Author}
%
\begin{Examples}
\begin{ExampleCode}
##---- Should be DIRECTLY executable !! ----
##-- ==>  Define data, use random,
##--	or do  help(data=index)  for the standard data sets.

\end{ExampleCode}
\end{Examples}
\inputencoding{utf8}
\HeaderA{retro.frq}{retro.frq.Rd}{retro.frq}
\keyword{\textbackslash{}textasciitilde{}kwd1}{retro.frq}
\keyword{\textbackslash{}textasciitilde{}kwd2}{retro.frq}
%
\begin{Description}\relax

Set a frq object up fo a retrospective analysis. Need some more testing. 
\end{Description}
%
\begin{Usage}
\begin{verbatim}
retro.frq(frq.obj, retro.tag.obj = NA)
\end{verbatim}
\end{Usage}
%
\begin{Arguments}
\begin{ldescription}
\item[\code{frq.obj}] 


\item[\code{retro.tag.obj}] 


\end{ldescription}
\end{Arguments}
%
\begin{Author}\relax

Simon Hoyle
\end{Author}
%
\begin{Examples}
\begin{ExampleCode}
##---- Should be DIRECTLY executable !! ----
##-- ==>  Define data, use random,
##--	or do  help(data=index)  for the standard data sets.

\end{ExampleCode}
\end{Examples}
\inputencoding{utf8}
\HeaderA{retro.tag}{retro.tag.Rd}{retro.tag}
\keyword{\textbackslash{}textasciitilde{}kwd1}{retro.tag}
\keyword{\textbackslash{}textasciitilde{}kwd2}{retro.tag}
%
\begin{Description}\relax

Set a tag object up fo a retrospective analysis. Need some more testing. 
\end{Description}
%
\begin{Usage}
\begin{verbatim}
retro.tag(tag.obj, yr)
\end{verbatim}
\end{Usage}
%
\begin{Arguments}
\begin{ldescription}
\item[\code{tag.obj}] 


\item[\code{yr}] 


\end{ldescription}
\end{Arguments}
%
\begin{Author}\relax

Simon Hoyle
\end{Author}
%
\begin{Examples}
\begin{ExampleCode}
##---- Should be DIRECTLY executable !! ----
##-- ==>  Define data, use random,
##--	or do  help(data=index)  for the standard data sets.

\end{ExampleCode}
\end{Examples}
\inputencoding{utf8}
\HeaderA{rm\_fisheries.doitall}{rm\_fisheries.doitall.Rd}{rm.Rul.fisheries.doitall}
\keyword{\textbackslash{}textasciitilde{}kwd1}{rm\_fisheries.doitall}
\keyword{\textbackslash{}textasciitilde{}kwd2}{rm\_fisheries.doitall}
%
\begin{Description}\relax

Removes all flags for specified fisheries from the doitall. 
\end{Description}
%
\begin{Usage}
\begin{verbatim}
rm_fisheries.doitall(a, rmfisheries)
\end{verbatim}
\end{Usage}
%
\begin{Arguments}
\begin{ldescription}
\item[\code{a}] 


\item[\code{rmfisheries}] 


\end{ldescription}
\end{Arguments}
%
\begin{Author}\relax

Simon Hoyle
\end{Author}
%
\begin{Examples}
\begin{ExampleCode}
##---- Should be DIRECTLY executable !! ----
##-- ==>  Define data, use random,
##--	or do  help(data=index)  for the standard data sets.

\end{ExampleCode}
\end{Examples}
\inputencoding{utf8}
\HeaderA{rm\_fisheries.frq}{rm\_fisheries.frq.Rd}{rm.Rul.fisheries.frq}
\keyword{\textbackslash{}textasciitilde{}kwd1}{rm\_fisheries.frq}
\keyword{\textbackslash{}textasciitilde{}kwd2}{rm\_fisheries.frq}
%
\begin{Description}\relax

Removes all catch and effort in specific fisheries. 
\end{Description}
%
\begin{Usage}
\begin{verbatim}
rm_fisheries.frq(frq.obj, fishery)
\end{verbatim}
\end{Usage}
%
\begin{Arguments}
\begin{ldescription}
\item[\code{frq.obj}] 


\item[\code{fishery}] 


\end{ldescription}
\end{Arguments}
%
\begin{Author}\relax

Simon Hoyle
\end{Author}
%
\begin{Examples}
\begin{ExampleCode}
##---- Should be DIRECTLY executable !! ----
##-- ==>  Define data, use random,
##--	or do  help(data=index)  for the standard data sets.

\end{ExampleCode}
\end{Examples}
\inputencoding{utf8}
\HeaderA{rm\_fisheries.tag}{rm\_fisheries.tag.Rd}{rm.Rul.fisheries.tag}
\keyword{\textbackslash{}textasciitilde{}kwd1}{rm\_fisheries.tag}
\keyword{\textbackslash{}textasciitilde{}kwd2}{rm\_fisheries.tag}
%
\begin{Description}\relax

Removes all recoveries in specified fisheries from a tag object. 
\end{Description}
%
\begin{Usage}
\begin{verbatim}
rm_fisheries.tag(tag.obj, fisheries)
\end{verbatim}
\end{Usage}
%
\begin{Arguments}
\begin{ldescription}
\item[\code{tag.obj}] 


\item[\code{fisheries}] 


\end{ldescription}
\end{Arguments}
%
\begin{Author}\relax

Simon Hoyle
\end{Author}
%
\begin{Examples}
\begin{ExampleCode}
##---- Should be DIRECTLY executable !! ----
##-- ==>  Define data, use random,
##--	or do  help(data=index)  for the standard data sets.

\end{ExampleCode}
\end{Examples}
\inputencoding{utf8}
\HeaderA{rm\_fishflag}{rm\_fishflag.Rd}{rm.Rul.fishflag}
\keyword{\textbackslash{}textasciitilde{}kwd1}{rm\_fishflag}
\keyword{\textbackslash{}textasciitilde{}kwd2}{rm\_fishflag}
%
\begin{Description}\relax

Removes all occurrences of changes to a specified fish flag from the doitall. 
\end{Description}
%
\begin{Usage}
\begin{verbatim}
rm_fishflag(doitall, flag)
\end{verbatim}
\end{Usage}
%
\begin{Arguments}
\begin{ldescription}
\item[\code{doitall}] 


\item[\code{flag}] 


\end{ldescription}
\end{Arguments}
%
\begin{Author}\relax

Simon Hoyle
\end{Author}
%
\begin{Examples}
\begin{ExampleCode}
##---- Should be DIRECTLY executable !! ----
##-- ==>  Define data, use random,
##--	or do  help(data=index)  for the standard data sets.

\end{ExampleCode}
\end{Examples}
\inputencoding{utf8}
\HeaderA{rm\_flag.doitall}{rm\_flag.doitall.Rd}{rm.Rul.flag.doitall}
\keyword{\textbackslash{}textasciitilde{}kwd1}{rm\_flag.doitall}
\keyword{\textbackslash{}textasciitilde{}kwd2}{rm\_flag.doitall}
%
\begin{Description}\relax

Removes all occurrences of changes to a specified flag from the doitall.
\end{Description}
%
\begin{Usage}
\begin{verbatim}
rm_flag.doitall(a, flagtype, flag, value)
\end{verbatim}
\end{Usage}
%
\begin{Arguments}
\begin{ldescription}
\item[\code{a}] 


\item[\code{flagtype}] 


\item[\code{flag}] 


\item[\code{value}] 


\end{ldescription}
\end{Arguments}
%
\begin{Author}\relax

Simon Hoyle
\end{Author}
%
\begin{Examples}
\begin{ExampleCode}
##---- Should be DIRECTLY executable !! ----
##-- ==>  Define data, use random,
##--	or do  help(data=index)  for the standard data sets.

\end{ExampleCode}
\end{Examples}
\inputencoding{utf8}
\HeaderA{run.profile}{run.profile.Rd}{run.profile}
\keyword{\textbackslash{}textasciitilde{}kwd1}{run.profile}
\keyword{\textbackslash{}textasciitilde{}kwd2}{run.profile}
%
\begin{Description}\relax

Run a likelihood profile analysis on a stock assessment. Needs to be generalized and tested. 
\end{Description}
%
\begin{Usage}
\begin{verbatim}
run.profile(rundir, rungrp, startpar = NA, ptype = "Fmult", target, nsteps = 300, penalty = 5e+05)
\end{verbatim}
\end{Usage}
%
\begin{Arguments}
\begin{ldescription}
\item[\code{rundir}] 


\item[\code{rungrp}] 


\item[\code{startpar}] 


\item[\code{ptype}] 


\item[\code{target}] 


\item[\code{nsteps}] 


\item[\code{penalty}] 


\end{ldescription}
\end{Arguments}
%
\begin{Author}\relax

Simon Hoyle
\end{Author}
%
\begin{Examples}
\begin{ExampleCode}
##---- Should be DIRECTLY executable !! ----
##-- ==>  Define data, use random,
##--	or do  help(data=index)  for the standard data sets.

\end{ExampleCode}
\end{Examples}
\inputencoding{utf8}
\HeaderA{seas.flag}{seas.flag.Rd}{seas.flag}
\keyword{\textbackslash{}textasciitilde{}kwd1}{seas.flag}
\keyword{\textbackslash{}textasciitilde{}kwd2}{seas.flag}
%
\begin{Description}\relax

Change the doitall object so that the specified fishery is made seasonal. 
\end{Description}
%
\begin{Usage}
\begin{verbatim}
seas.flag(a, fishery, flagnum, seasf.list)
\end{verbatim}
\end{Usage}
%
\begin{Arguments}
\begin{ldescription}
\item[\code{a}] 


\item[\code{fishery}] 


\item[\code{flagnum}] 


\item[\code{seasf.list}] 


\end{ldescription}
\end{Arguments}
%
\begin{Author}\relax

Simon Hoyle
\end{Author}
%
\begin{Examples}
\begin{ExampleCode}
##---- Should be DIRECTLY executable !! ----
##-- ==>  Define data, use random,
##--	or do  help(data=index)  for the standard data sets.

\end{ExampleCode}
\end{Examples}
\inputencoding{utf8}
\HeaderA{seas.frq}{seas.frq.Rd}{seas.frq}
\keyword{\textbackslash{}textasciitilde{}kwd1}{seas.frq}
\keyword{\textbackslash{}textasciitilde{}kwd2}{seas.frq}
%
\begin{Description}\relax

Change the frq object so that the specified fishery is made seasonal. 
\end{Description}
%
\begin{Usage}
\begin{verbatim}
seas.frq(frq.obj, seas.fish)
\end{verbatim}
\end{Usage}
%
\begin{Arguments}
\begin{ldescription}
\item[\code{frq.obj}] 


\item[\code{seas.fish}] 


\end{ldescription}
\end{Arguments}
%
\begin{Author}\relax

Simon Hoyle
\end{Author}
%
\begin{Examples}
\begin{ExampleCode}
##---- Should be DIRECTLY executable !! ----
##-- ==>  Define data, use random,
##--	or do  help(data=index)  for the standard data sets.

\end{ExampleCode}
\end{Examples}
\inputencoding{utf8}
\HeaderA{seas.tag}{seas.tag.Rd}{seas.tag}
\keyword{\textbackslash{}textasciitilde{}kwd1}{seas.tag}
\keyword{\textbackslash{}textasciitilde{}kwd2}{seas.tag}
%
\begin{Description}\relax

Change the tag object so that the specified fishery is made seasonal. 
\end{Description}
%
\begin{Usage}
\begin{verbatim}
seas.tag(tag.obj, fishlist)
\end{verbatim}
\end{Usage}
%
\begin{Arguments}
\begin{ldescription}
\item[\code{tag.obj}] 


\item[\code{fishlist}] 


\end{ldescription}
\end{Arguments}
%
\begin{Author}\relax

Simon Hoyle
\end{Author}
%
\begin{Examples}
\begin{ExampleCode}
##---- Should be DIRECTLY executable !! ----
##-- ==>  Define data, use random,
##--	or do  help(data=index)  for the standard data sets.

\end{ExampleCode}
\end{Examples}
\inputencoding{utf8}
\HeaderA{setup.cpue}{setup.cpue.Rd}{setup.cpue}
\keyword{\textbackslash{}textasciitilde{}kwd1}{setup.cpue}
\keyword{\textbackslash{}textasciitilde{}kwd2}{setup.cpue}
%
\begin{Description}\relax

Replace particular CPUE series with other values, which are supplied. 
The setup files are generally used in structural sensitivity analyses. They modify an object that contains all the MULTIFAN-CL input files. 
\end{Description}
%
\begin{Usage}
\begin{verbatim}
setup.cpue(rungrp, sourcedir, cpue, spp)
\end{verbatim}
\end{Usage}
%
\begin{Arguments}
\begin{ldescription}
\item[\code{rungrp}] 


\item[\code{sourcedir}] 


\item[\code{cpue}] 


\item[\code{spp}] 


\end{ldescription}
\end{Arguments}
%
\begin{Author}\relax

Simon Hoyle
\end{Author}
%
\begin{Examples}
\begin{ExampleCode}
##---- Should be DIRECTLY executable !! ----
##-- ==>  Define data, use random,
##--	or do  help(data=index)  for the standard data sets.

\end{ExampleCode}
\end{Examples}
\inputencoding{utf8}
\HeaderA{setup.effcreep}{setup.effcreep.Rd}{setup.effcreep}
\keyword{\textbackslash{}textasciitilde{}kwd1}{setup.effcreep}
\keyword{\textbackslash{}textasciitilde{}kwd2}{setup.effcreep}
%
\begin{Description}\relax

Adjust the effort in specified fisheries to adjust for a steady increase in fishing power at a specified rate. 
The setup files are generally used in structural sensitivity analyses. They modify an object that contains all the MULTIFAN-CL input files. 
\end{Description}
%
\begin{Usage}
\begin{verbatim}
setup.effcreep(rungrp, creeprate)
\end{verbatim}
\end{Usage}
%
\begin{Arguments}
\begin{ldescription}
\item[\code{rungrp}] 


\item[\code{creeprate}] 


\end{ldescription}
\end{Arguments}
%
\begin{Author}\relax

Simon Hoyle
\end{Author}
%
\begin{Examples}
\begin{ExampleCode}
##---- Should be DIRECTLY executable !! ----
##-- ==>  Define data, use random,
##--	or do  help(data=index)  for the standard data sets.

\end{ExampleCode}
\end{Examples}
\inputencoding{utf8}
\HeaderA{setup.growth}{setup.growth.Rd}{setup.growth}
\keyword{\textbackslash{}textasciitilde{}kwd1}{setup.growth}
\keyword{\textbackslash{}textasciitilde{}kwd2}{setup.growth}
%
\begin{Description}\relax

Change the growth parameters to the values supplied in VBopt. 
The setup files are generally used in structural sensitivity analyses. They modify an object that contains all the MULTIFAN-CL input files. 
\end{Description}
%
\begin{Usage}
\begin{verbatim}
setup.growth(rungrp, VBopt)
\end{verbatim}
\end{Usage}
%
\begin{Arguments}
\begin{ldescription}
\item[\code{rungrp}] 


\item[\code{VBopt}] 


\end{ldescription}
\end{Arguments}
%
\begin{Author}\relax

Simon Hoyle
\end{Author}
%
\begin{Examples}
\begin{ExampleCode}
##---- Should be DIRECTLY executable !! ----
##-- ==>  Define data, use random,
##--	or do  help(data=index)  for the standard data sets.

\end{ExampleCode}
\end{Examples}
\inputencoding{utf8}
\HeaderA{setup.growth.offsets}{setup.growth.offsets.Rd}{setup.growth.offsets}
\keyword{\textbackslash{}textasciitilde{}kwd1}{setup.growth.offsets}
\keyword{\textbackslash{}textasciitilde{}kwd2}{setup.growth.offsets}
%
\begin{Description}\relax

Modifies the growth offests to the specified values, and turns on their use and estimation in a specified phase. 
The setup files are generally used in structural sensitivity analyses. They modify an object that contains all the MULTIFAN-CL input files. 
\end{Description}
%
\begin{Usage}
\begin{verbatim}
setup.growth.offsets(rungrp, ageclasses, penwt, phase, tog)
\end{verbatim}
\end{Usage}
%
\begin{Arguments}
\begin{ldescription}
\item[\code{rungrp}] 


\item[\code{ageclasses}] 


\item[\code{penwt}] 


\item[\code{phase}] 


\item[\code{tog}] 


\end{ldescription}
\end{Arguments}
%
\begin{Author}\relax

Simon Hoyle
\end{Author}
%
\begin{Examples}
\begin{ExampleCode}
##---- Should be DIRECTLY executable !! ----
##-- ==>  Define data, use random,
##--	or do  help(data=index)  for the standard data sets.

\end{ExampleCode}
\end{Examples}
\inputencoding{utf8}
\HeaderA{setup.idphcatch}{setup.idphcatch.Rd}{setup.idphcatch}
\keyword{\textbackslash{}textasciitilde{}kwd1}{setup.idphcatch}
\keyword{\textbackslash{}textasciitilde{}kwd2}{setup.idphcatch}
%
\begin{Description}\relax

Replace the catches in Indonesia Phillippines fisheries with values supplied in a folder with prefix 'idph'. Needs modification to be more general. 
The setup files are generally used in structural sensitivity analyses. They modify an object that contains all the MULTIFAN-CL input files. 
\end{Description}
%
\begin{Usage}
\begin{verbatim}
setup.idphcatch(rungrp, sourcedir, idph, spp)
\end{verbatim}
\end{Usage}
%
\begin{Arguments}
\begin{ldescription}
\item[\code{rungrp}] 


\item[\code{sourcedir}] 


\item[\code{idph}] 


\item[\code{spp}] 


\end{ldescription}
\end{Arguments}
%
\begin{Author}\relax

Nick Davies
\end{Author}
%
\begin{Examples}
\begin{ExampleCode}
##---- Should be DIRECTLY executable !! ----
##-- ==>  Define data, use random,
##--	or do  help(data=index)  for the standard data sets.

\end{ExampleCode}
\end{Examples}
\inputencoding{utf8}
\HeaderA{setup.lensel}{setup.lensel.Rd}{setup.lensel}
\keyword{\textbackslash{}textasciitilde{}kwd1}{setup.lensel}
\keyword{\textbackslash{}textasciitilde{}kwd2}{setup.lensel}
%
\begin{Description}\relax

Change selectivity to fully length-based in the specified fisheries. 
The setup files are generally used in structural sensitivity analyses. They modify an object that contains all the MULTIFAN-CL input files. 
\end{Description}
%
\begin{Usage}
\begin{verbatim}
setup.lensel(rungrp, fisheries, tog)
\end{verbatim}
\end{Usage}
%
\begin{Arguments}
\begin{ldescription}
\item[\code{rungrp}] 


\item[\code{fisheries}] 


\item[\code{tog}] 


\end{ldescription}
\end{Arguments}
%
\begin{Author}\relax

Simon Hoyle
\end{Author}
%
\begin{Examples}
\begin{ExampleCode}
##---- Should be DIRECTLY executable !! ----
##-- ==>  Define data, use random,
##--	or do  help(data=index)  for the standard data sets.

\end{ExampleCode}
\end{Examples}
\inputencoding{utf8}
\HeaderA{setup.LFwt}{setup.LFwt.Rd}{setup.LFwt}
\keyword{\textbackslash{}textasciitilde{}kwd1}{setup.LFwt}
\keyword{\textbackslash{}textasciitilde{}kwd2}{setup.LFwt}
%
\begin{Description}\relax

Change the likelihood weight on the length frequencies to the specified value in specified fisheries, defaulting to all fisheries. 
The setup files are generally used in structural sensitivity analyses. They modify an object that contains all the MULTIFAN-CL input files. 
\end{Description}
%
\begin{Usage}
\begin{verbatim}
setup.LFwt(rungrp, newLFwt)
\end{verbatim}
\end{Usage}
%
\begin{Arguments}
\begin{ldescription}
\item[\code{rungrp}] 


\item[\code{newLFwt}] 


\end{ldescription}
\end{Arguments}
%
\begin{Author}\relax

Simon Hoyle
\end{Author}
%
\begin{Examples}
\begin{ExampleCode}
##---- Should be DIRECTLY executable !! ----
##-- ==>  Define data, use random,
##--	or do  help(data=index)  for the standard data sets.

\end{ExampleCode}
\end{Examples}
\inputencoding{utf8}
\HeaderA{setup.M}{setup.M.Rd}{setup.M}
\keyword{\textbackslash{}textasciitilde{}kwd1}{setup.M}
\keyword{\textbackslash{}textasciitilde{}kwd2}{setup.M}
%
\begin{Description}\relax

Change the starting value of mean natural mortality in the ini file, and turn off M estimation. 
The setup files are generally used in structural sensitivity analyses. They modify an object that contains all the MULTIFAN-CL input files. 
\end{Description}
%
\begin{Usage}
\begin{verbatim}
setup.M(rungrp, newM)
\end{verbatim}
\end{Usage}
%
\begin{Arguments}
\begin{ldescription}
\item[\code{rungrp}] 


\item[\code{newM}] 


\end{ldescription}
\end{Arguments}
%
\begin{Author}\relax

Simon Hoyle
\end{Author}
%
\begin{Examples}
\begin{ExampleCode}
##---- Should be DIRECTLY executable !! ----
##-- ==>  Define data, use random,
##--	or do  help(data=index)  for the standard data sets.

\end{ExampleCode}
\end{Examples}
\inputencoding{utf8}
\HeaderA{setup.pscatch}{setup.pscatch.Rd}{setup.pscatch}
\keyword{\textbackslash{}textasciitilde{}kwd1}{setup.pscatch}
\keyword{\textbackslash{}textasciitilde{}kwd2}{setup.pscatch}
%
\begin{Description}\relax

Replace the catches in purse seine fisheries (2011 WCPO bigeye) with values supplied in a folder with prefix 'PScatch'. Needs modification to be more general. 
The setup files are generally used in structural sensitivity analyses. They modify an object that contains all the MULTIFAN-CL input files. 
\end{Description}
%
\begin{Usage}
\begin{verbatim}
setup.pscatch(rungrp, sourcedir, PScatch, spp)
\end{verbatim}
\end{Usage}
%
\begin{Arguments}
\begin{ldescription}
\item[\code{rungrp}] 


\item[\code{sourcedir}] 


\item[\code{PScatch}] 


\item[\code{spp}] 


\end{ldescription}
\end{Arguments}
%
\begin{Author}\relax

Nick Davies
\end{Author}
%
\begin{Examples}
\begin{ExampleCode}
##---- Should be DIRECTLY executable !! ----
##-- ==>  Define data, use random,
##--	or do  help(data=index)  for the standard data sets.

\end{ExampleCode}
\end{Examples}
\inputencoding{utf8}
\HeaderA{setup.startyr}{setup.startyr.Rd}{setup.startyr}
\keyword{\textbackslash{}textasciitilde{}kwd1}{setup.startyr}
\keyword{\textbackslash{}textasciitilde{}kwd2}{setup.startyr}
%
\begin{Description}\relax

Changes the start year of the assessment. Runs start\_year.frq. Currently doesn't change the tag file. 
The setup files are generally used in structural sensitivity analyses. They modify an object that contains all the MULTIFAN-CL input files. 
\end{Description}
%
\begin{Usage}
\begin{verbatim}
setup.startyr(rungrp, newstartyr)
\end{verbatim}
\end{Usage}
%
\begin{Arguments}
\begin{ldescription}
\item[\code{rungrp}] 


\item[\code{newstartyr}] 


\end{ldescription}
\end{Arguments}
%
\begin{Author}\relax

Simon Hoyle
\end{Author}
%
\begin{Examples}
\begin{ExampleCode}
##---- Should be DIRECTLY executable !! ----
##-- ==>  Define data, use random,
##--	or do  help(data=index)  for the standard data sets.

\end{ExampleCode}
\end{Examples}
\inputencoding{utf8}
\HeaderA{setup.steepness}{setup.steepness.Rd}{setup.steepness}
\keyword{\textbackslash{}textasciitilde{}kwd1}{setup.steepness}
\keyword{\textbackslash{}textasciitilde{}kwd2}{setup.steepness}
%
\begin{Description}\relax

Changes the fixed value of steepness in the assessment by editing the doitall file. 
The setup files are generally used in structural sensitivity analyses. They modify an object that contains all the MULTIFAN-CL input files. 
\end{Description}
%
\begin{Usage}
\begin{verbatim}
setup.steepness(rungrp, newsteep)
\end{verbatim}
\end{Usage}
%
\begin{Arguments}
\begin{ldescription}
\item[\code{rungrp}] 


\item[\code{newsteep}] 


\end{ldescription}
\end{Arguments}
%
\begin{Author}\relax

Simon Hoyle
\end{Author}
%
\begin{Examples}
\begin{ExampleCode}
##---- Should be DIRECTLY executable !! ----
##-- ==>  Define data, use random,
##--	or do  help(data=index)  for the standard data sets.

\end{ExampleCode}
\end{Examples}
\inputencoding{utf8}
\HeaderA{setup.timesplit}{setup.timesplit.Rd}{setup.timesplit}
\keyword{\textbackslash{}textasciitilde{}kwd1}{setup.timesplit}
\keyword{\textbackslash{}textasciitilde{}kwd2}{setup.timesplit}
%
\begin{Description}\relax

Modifies the assessment files to include a time split, defined by the parameter splitx. 
The setup files are generally used in structural sensitivity analyses. They modify an object that contains all the MULTIFAN-CL input files. 
\end{Description}
%
\begin{Usage}
\begin{verbatim}
setup.timesplit(rungrp, splitx, storefish)
\end{verbatim}
\end{Usage}
%
\begin{Arguments}
\begin{ldescription}
\item[\code{rungrp}] 


\item[\code{splitx}] 


\item[\code{storefish}] 


\end{ldescription}
\end{Arguments}
%
\begin{Author}\relax

Simon Hoyle
\end{Author}
%
\begin{Examples}
\begin{ExampleCode}
##---- Should be DIRECTLY executable !! ----
##-- ==>  Define data, use random,
##--	or do  help(data=index)  for the standard data sets.

\end{ExampleCode}
\end{Examples}
\inputencoding{utf8}
\HeaderA{sort.frq}{sort.frq.Rd}{sort.frq}
\keyword{\textbackslash{}textasciitilde{}kwd1}{sort.frq}
\keyword{\textbackslash{}textasciitilde{}kwd2}{sort.frq}
%
\begin{Description}\relax

Sorts the data in a frq file with the fisheries and times in ascending order. 
\end{Description}
%
\begin{Usage}
\begin{verbatim}
sort.frq(frq.obj)
\end{verbatim}
\end{Usage}
%
\begin{Arguments}
\begin{ldescription}
\item[\code{frq.obj}] 


\end{ldescription}
\end{Arguments}
%
\begin{Author}\relax

Simon Hoyle
\end{Author}
%
\begin{Examples}
\begin{ExampleCode}
##---- Should be DIRECTLY executable !! ----
##-- ==>  Define data, use random,
##--	or do  help(data=index)  for the standard data sets.

\end{ExampleCode}
\end{Examples}
\inputencoding{utf8}
\HeaderA{start\_year.frq}{start\_year.frq.Rd}{start.Rul.year.frq}
\keyword{\textbackslash{}textasciitilde{}kwd1}{start\_year.frq}
\keyword{\textbackslash{}textasciitilde{}kwd2}{start\_year.frq}
%
\begin{Description}\relax

Change the starting year of the assessment by removing all frq data before that time and changing the start year parameter. 
\end{Description}
%
\begin{Usage}
\begin{verbatim}
start_year.frq(frq.obj, start_yr, halfyr = F)
\end{verbatim}
\end{Usage}
%
\begin{Arguments}
\begin{ldescription}
\item[\code{frq.obj}] 


\item[\code{start\_yr}] 


\item[\code{halfyr}] 


\end{ldescription}
\end{Arguments}
%
\begin{Author}\relax

Simon Hoyle
\end{Author}
%
\begin{Examples}
\begin{ExampleCode}
##---- Should be DIRECTLY executable !! ----
##-- ==>  Define data, use random,
##--	or do  help(data=index)  for the standard data sets.

\end{ExampleCode}
\end{Examples}
\inputencoding{utf8}
\HeaderA{steepness.doit}{steepness.doit.Rd}{steepness.doit}
\keyword{\textbackslash{}textasciitilde{}kwd1}{steepness.doit}
\keyword{\textbackslash{}textasciitilde{}kwd2}{steepness.doit}
%
\begin{Description}\relax

Inserts a new line "recruitmentConstraints 01.par \#\#\#" after PHASE 1. 
This allows steepness to be fixed at a chosen level. 

\end{Description}
%
\begin{Usage}
\begin{verbatim}
steepness.doit(doitall, new.steepness, add_header = T, gap = 2)
\end{verbatim}
\end{Usage}
%
\begin{Arguments}
\begin{ldescription}
\item[\code{doitall}] 


\item[\code{new.steepness}] 


\item[\code{add\_header}] 


\item[\code{gap}] 


\end{ldescription}
\end{Arguments}
%
\begin{Author}\relax

Simon Hoyle
\end{Author}
%
\begin{Examples}
\begin{ExampleCode}
##---- Should be DIRECTLY executable !! ----
##-- ==>  Define data, use random,
##--	or do  help(data=index)  for the standard data sets.

\end{ExampleCode}
\end{Examples}
\inputencoding{utf8}
\HeaderA{summarise.size.frq.bet}{summarise.size.frq.bet.Rd}{summarise.size.frq.bet}
\keyword{\textbackslash{}textasciitilde{}kwd1}{summarise.size.frq.bet}
\keyword{\textbackslash{}textasciitilde{}kwd2}{summarise.size.frq.bet}
%
\begin{Description}\relax

Takes two frq files from the bigeye assessment and compares the length and weight data on an annual basis for the range of years
Just does the last 15 years at the moment. Specific for the 2009 BET assessment, and included as an example. . 
\end{Description}
%
\begin{Usage}
\begin{verbatim}
summarise.size.frq.bet(frq1, fishery = 5)
\end{verbatim}
\end{Usage}
%
\begin{Arguments}
\begin{ldescription}
\item[\code{frq1}] 


\item[\code{fishery}] 


\end{ldescription}
\end{Arguments}
%
\begin{Author}\relax

Shelton Harley
\end{Author}
%
\begin{Examples}
\begin{ExampleCode}
##---- Should be DIRECTLY executable !! ----
##-- ==>  Define data, use random,
##--	or do  help(data=index)  for the standard data sets.

\end{ExampleCode}
\end{Examples}
\inputencoding{utf8}
\HeaderA{tag\_grps\_rm}{tag\_grps\_rm.Rd}{tag.Rul.grps.Rul.rm}
\keyword{\textbackslash{}textasciitilde{}kwd1}{tag\_grps\_rm}
\keyword{\textbackslash{}textasciitilde{}kwd2}{tag\_grps\_rm}
%
\begin{Description}\relax

Remove the specified tag groups from the tag object. 
\end{Description}
%
\begin{Usage}
\begin{verbatim}
tag_grps_rm(tag.obj, keep)
\end{verbatim}
\end{Usage}
%
\begin{Arguments}
\begin{ldescription}
\item[\code{tag.obj}] 


\item[\code{keep}] 


\end{ldescription}
\end{Arguments}
%
\begin{Author}\relax

Simon Hoyle
\end{Author}
%
\begin{Examples}
\begin{ExampleCode}
##---- Should be DIRECTLY executable !! ----
##-- ==>  Define data, use random,
##--	or do  help(data=index)  for the standard data sets.

\end{ExampleCode}
\end{Examples}
\inputencoding{utf8}
\HeaderA{timesplit.doitall}{timesplit.doitall.Rd}{timesplit.doitall}
\keyword{\textbackslash{}textasciitilde{}kwd1}{timesplit.doitall}
\keyword{\textbackslash{}textasciitilde{}kwd2}{timesplit.doitall}
%
\begin{Description}\relax

Changes a doitall file to account for time splits. 
Time splits occur when a fishery is broken up into several fisheries by time, with dates and new fishery codes specified in the 'fishsplit' parameter. 
\end{Description}
%
\begin{Usage}
\begin{verbatim}
timesplit.doitall(doitall, fishsplit, qsplit = T)
\end{verbatim}
\end{Usage}
%
\begin{Arguments}
\begin{ldescription}
\item[\code{doitall}] 


\item[\code{fishsplit}] 


\item[\code{qsplit}] 


\end{ldescription}
\end{Arguments}
%
\begin{Author}\relax

Simon Hoyle
\end{Author}
%
\begin{Examples}
\begin{ExampleCode}
##---- Should be DIRECTLY executable !! ----
##-- ==>  Define data, use random,
##--	or do  help(data=index)  for the standard data sets.

\end{ExampleCode}
\end{Examples}
\inputencoding{utf8}
\HeaderA{timesplit.frq}{timesplit.frq.Rd}{timesplit.frq}
\keyword{\textbackslash{}textasciitilde{}kwd1}{timesplit.frq}
\keyword{\textbackslash{}textasciitilde{}kwd2}{timesplit.frq}
%
\begin{Description}\relax

Changes a frq file to account for time splits. 
Time splits occur when a fishery is broken up into several fisheries by time, with dates and new fishery codes specified in the 'fishsplit' parameter. 
\end{Description}
%
\begin{Usage}
\begin{verbatim}
timesplit.frq(frq.obj, divyrs, div.fish)
\end{verbatim}
\end{Usage}
%
\begin{Arguments}
\begin{ldescription}
\item[\code{frq.obj}] 


\item[\code{divyrs}] 


\item[\code{div.fish}] 


\end{ldescription}
\end{Arguments}
%
\begin{Author}\relax

Simon Hoyle
\end{Author}
%
\begin{Examples}
\begin{ExampleCode}
##---- Should be DIRECTLY executable !! ----
##-- ==>  Define data, use random,
##--	or do  help(data=index)  for the standard data sets.

\end{ExampleCode}
\end{Examples}
\inputencoding{utf8}
\HeaderA{timesplit.tag}{timesplit.tag.Rd}{timesplit.tag}
\keyword{\textbackslash{}textasciitilde{}kwd1}{timesplit.tag}
\keyword{\textbackslash{}textasciitilde{}kwd2}{timesplit.tag}
%
\begin{Description}\relax

Changes a tag file to account for time splits. 
Time splits occur when a fishery is broken up into several fisheries by time, with dates and new fishery codes specified in the 'fishsplit' parameter. 
\end{Description}
%
\begin{Usage}
\begin{verbatim}
timesplit.tag(tag.obj, fishsplit)
\end{verbatim}
\end{Usage}
%
\begin{Arguments}
\begin{ldescription}
\item[\code{tag.obj}] 


\item[\code{fishsplit}] 


\end{ldescription}
\end{Arguments}
%
\begin{Author}\relax

Simon Hoyle
\end{Author}
%
\begin{Examples}
\begin{ExampleCode}
##---- Should be DIRECTLY executable !! ----
##-- ==>  Define data, use random,
##--	or do  help(data=index)  for the standard data sets.

\end{ExampleCode}
\end{Examples}
\inputencoding{utf8}
\HeaderA{varfromstr}{varfromstr.Rd}{varfromstr}
\keyword{\textbackslash{}textasciitilde{}kwd1}{varfromstr}
\keyword{\textbackslash{}textasciitilde{}kwd2}{varfromstr}
%
\begin{Usage}
\begin{verbatim}
varfromstr(datstring, cols = c(2:3))
\end{verbatim}
\end{Usage}
%
\begin{Arguments}
\begin{ldescription}
\item[\code{datstring}] 


\item[\code{cols}] 


\end{ldescription}
\end{Arguments}
%
\begin{Author}\relax

Pierre Kleiber
\end{Author}
%
\begin{Examples}
\begin{ExampleCode}
##---- Should be DIRECTLY executable !! ----
##-- ==>  Define data, use random,
##--	or do  help(data=index)  for the standard data sets.

\end{ExampleCode}
\end{Examples}
\inputencoding{utf8}
\HeaderA{write.frq}{write.frq.Rd}{write.frq}
\keyword{\textbackslash{}textasciitilde{}kwd1}{write.frq}
\keyword{\textbackslash{}textasciitilde{}kwd2}{write.frq}
%
\begin{Description}\relax

Writes out the frq file (catch and effort, size frequency and model structure). 
\end{Description}
%
\begin{Usage}
\begin{verbatim}
write.frq(frqfile, frq.obj)
\end{verbatim}
\end{Usage}
%
\begin{Arguments}
\begin{ldescription}
\item[\code{frqfile}] 


\item[\code{frq.obj}] 


\end{ldescription}
\end{Arguments}
%
\begin{Author}\relax

Simon Hoyle
\end{Author}
%
\begin{Examples}
\begin{ExampleCode}
##---- Should be DIRECTLY executable !! ----
##-- ==>  Define data, use random,
##--	or do  help(data=index)  for the standard data sets.

\end{ExampleCode}
\end{Examples}
\inputencoding{utf8}
\HeaderA{write.ini}{write.ini.Rd}{write.ini}
\keyword{\textbackslash{}textasciitilde{}kwd1}{write.ini}
\keyword{\textbackslash{}textasciitilde{}kwd2}{write.ini}
%
\begin{Description}\relax

Writes out the ini file in MULTIFAN-CL inoput format, from an ini object. 
\end{Description}
%
\begin{Usage}
\begin{verbatim}
write.ini(ini.file, ini.obj, old.format=FALSE)
\end{verbatim}
\end{Usage}
%
\begin{Arguments}
\begin{ldescription}
\item[\code{ini.file}] 


\item[\code{ini.obj}] 


\item[\code{old.format}] 


\end{ldescription}
\end{Arguments}
%
\begin{Author}\relax

Simon Hoyle
\end{Author}
%
\begin{Examples}
\begin{ExampleCode}
##---- Should be DIRECTLY executable !! ----
##-- ==>  Define data, use random,
##--	or do  help(data=index)  for the standard data sets.

\end{ExampleCode}
\end{Examples}
\inputencoding{utf8}
\HeaderA{write.par}{write.par.Rd}{write.par}
\keyword{\textbackslash{}textasciitilde{}kwd1}{write.par}
\keyword{\textbackslash{}textasciitilde{}kwd2}{write.par}
%
\begin{Description}\relax

Writes out the par file from a par object. 
\end{Description}
%
\begin{Usage}
\begin{verbatim}
write.par(par.file, par.obj)
\end{verbatim}
\end{Usage}
%
\begin{Arguments}
\begin{ldescription}
\item[\code{par.file}] 


\item[\code{par.obj}] 


\end{ldescription}
\end{Arguments}
%
\begin{Author}\relax

Simon Hoyle
\end{Author}
%
\begin{Examples}
\begin{ExampleCode}
##---- Should be DIRECTLY executable !! ----
##-- ==>  Define data, use random,
##--	or do  help(data=index)  for the standard data sets.

\end{ExampleCode}
\end{Examples}
\inputencoding{utf8}
\HeaderA{write.tag}{write.tag.Rd}{write.tag}
\keyword{\textbackslash{}textasciitilde{}kwd1}{write.tag}
\keyword{\textbackslash{}textasciitilde{}kwd2}{write.tag}
%
\begin{Description}\relax

Writes a tag object out into a *.tag text file for input to MULTIFAN-CL. 
\end{Description}
%
\begin{Usage}
\begin{verbatim}
write.tag(tagfile, tag.obj)
\end{verbatim}
\end{Usage}
%
\begin{Arguments}
\begin{ldescription}
\item[\code{tagfile}] 


\item[\code{tag.obj}] 


\end{ldescription}
\end{Arguments}
%
\begin{Author}\relax

Simon Hoyle
\end{Author}
%
\begin{Examples}
\begin{ExampleCode}
##---- Should be DIRECTLY executable !! ----
##-- ==>  Define data, use random,
##--	or do  help(data=index)  for the standard data sets.

\end{ExampleCode}
\end{Examples}
\inputencoding{utf8}
\HeaderA{write\_nmd.frq}{write\_nmd.frq.Rd}{write.Rul.nmd.frq}
\keyword{\textbackslash{}textasciitilde{}kwd1}{write\_nmd.frq}
\keyword{\textbackslash{}textasciitilde{}kwd2}{write\_nmd.frq}
%
\begin{Description}\relax

An alternative write.frq function - writes out the frq file (catch and effort, size frequency and model structure)
\end{Description}
%
\begin{Usage}
\begin{verbatim}
write_nmd.frq(new.frq, frq.obj)
\end{verbatim}
\end{Usage}
%
\begin{Arguments}
\begin{ldescription}
\item[\code{new.frq}] 


\item[\code{frq.obj}] 


\end{ldescription}
\end{Arguments}
%
\begin{Author}\relax

Nick Davies
\end{Author}
%
\begin{Examples}
\begin{ExampleCode}
##---- Should be DIRECTLY executable !! ----
##-- ==>  Define data, use random,
##--	or do  help(data=index)  for the standard data sets.

\end{ExampleCode}
\end{Examples}
\printindex{}
\end{document}
